\documentclass{article}
\usepackage{color}

\begin{document}
\section*{madam}
no {\bf \color{red} madam} speaker i will not
\vspace{8mm}
i yield myself such time as i may consume {\bf \color{red} madam} speaker h r 1726 is a noncontroversial bill to reorganize the coast guard s permanent authorities under title 14 of the u s code in order to improve the clarity and the coherence of the title i would like to thank my assistant and counsel dave jansen and john rayfield on the other side for recognizing that there is confusion and so this legislation while it makes no substantive changes to the coast guard s existing authorities and policies under title 14 does save for some conforming changes to create a new chapter for all of the coast guard national advisory councils and execute a handful of other minor transfers it just simply ends a lot of confusion as people try to figure out where it is in the code that they must look accordingly i am comfortable with supporting this legislation although i do note that it will impose some additional administrative costs on the coast guard as they set about implementing it i want to commend the chairman mr hunter for the coast guard and maritime transportation subcommittee for his initiative and for his staff s initiative in taking on this important little bit of housekeeping that is going to make all of our lives easier i will certainly urge all of the members to support it i also appreciate chairman hunter s work with me in an effort to try to solve another problem this is only one piece of our effort to try to improve the maritime industry while we are trying to move h r 1726 today we have more work to do and to that end i look forward to working with chairman hunter on bringing h r 2593 to the floor before the august recess together with the full committee chairman bill shuster and ranking democrat peter defazio h r 2593 deals with the federal maritime commission authorization act of 2017 this legislation would update and strengthen the shipping act to address the current upheaval in the global shipping markets that affect u s foreign trade the formation of three large ocean carrier alliances has raised legitimate concerns among u s port service providers maritime terminal operators and tug operators by virtue of the sheer size and volume of trade that these alliances carry they will have a decided advantage in determining ports of call negotiating contracts and shifting costs all at the expense possibly of our domestic port service providers i know chairman hunter shares my concerns and neither of us are indifferent we are going to have to deal with this and that will be our next piece of legislation on the floor the chairman is on top of it with his staff and we look forward to that bill h r 2593 being reported out of the transportation committee as it clarifies the federal maritime commission s authority to oversee and aggressively deal with competition so what we have today is one of two pieces of legislation that we intend to move forward dealing with the overall well being of the maritime industry i highly recommend h r 1726 to the floor it is noncontroversial and should pass the other piece of legislation will be here shortly and we will take that up at that time {\bf \color{red} madam} speaker i yield back the balance of my time
\vspace{8mm}
my motion to recommit would open up the entire federal government so that the part time national guard not only receives their pay but also the full time national guard they would receive all of their benefits they would receive funds for the equipment to do their jobs there would no longer be furloughs can the chair explain why it is not germane to keep all of the needs of the national guard open for public service instead of only their pay if we re paying our national guard but they can t do their jobs what sense does that make are we asking our brave soldiers simply to sit at their desks what kind of strange house is this that would force that situation on our brave men and women the brave men and women that have been so remarkably addressed by the gentleman across the aisle {\bf \color{red} madam} speaker if you rule this motion out of order does that mean we will not have a chance to keep the entire federal government open today can the chair please explain why we can t keep our part time national guard and the entire federal government open today\pagebreak

\section*{consume}
mr chairman this young amendment looks eerily familiar to the so called secret science reform act h r 1030 that the house passed in a partisan vote back in march except the problem is that this bill is actually even worse h r 1030 would have applied these harmful restrictions to the epa but this amendment that we are looking at today would affect every single federal agency let s look the amendment would require an agency as part of its rulemaking process to make all information used in the creation of a rule publicly accessible including all of the data that would mean that any data that is considered confidential such as health information or business records would most likely become off limits so for example an agency trying to create labeling requirements for toxic chemicals wouldn t be able to use a study that uses personal health data as long as that data is deemed confidential new scientific methods and data could be restricted because the information includes data protected by intellectual property laws when we passed the secret science act on a partisan vote last march i mentioned in my opposition that it would force the epa to choose between protecting our health and environment and maintaining the privacy of patient medical records and the confidentiality of business records and if that argument isn t enough let s consider the costs when the house science committee was considering the bill that i mentioned previously the secret science act that does exactly the same thing that the young amendment does except to all federal agencies democrats on the committee pointed out that the congressional budget office estimated just for that one bill that it would cost the epa 250 million to comply with the new regulations if that is how much it is going to cost the epa for one regulatory requirement imagine what the cost would be if you expand this mandate across every single federal agency the cost would be astronomical between the cost and the harmful restrictions that this imposes on our federal agencies the amendment sets up an impossible hurdle for those agencies to overcome we are asking them to decide between compromising institutional review board ethics and doing their job to protect the american people it is very clear that the young amendment and provisions like it are not in fact about transparency it really is to block federal agencies from doing their jobs their jobs of protecting our air giving us clean water making sure that our food supply is safe checking on medical devices so that they don t harm us our prescription drugs so that they don t make us sick our privacy safeguards for our workplace information our workplace safety standards protections against wall street and its predatory lending practices i would ask my colleagues to oppose this harmful and antiscience amendment oppose the final bill and oppose this amendment because of the restrictions that it would place on the american people i yield such time as he may {\bf \color{red} consume} to the gentleman from georgia mr johnson
\vspace{8mm}
maloney of new york mr speaker i yield myself such time as i may {\bf \color{red} consume} mr speaker we had other speakers scheduled from new york but they are not on the floor now so i would just like to say in closing that this is critically important legislation i can speak from personal experience having represented new york during and after 9 11 that after 9 11 you could not even build a hot dog stand all construction stopped no one could get any insurance the only insurance available was from lloyds of london and it was incredibly expensive and people could not afford it we lost thousands and thousands of jobs and it happened also when we came together and started to rebuild not only in new york but the pentagon and pennsylvania i would say of all the programs that this body put forward and there were many and i thank my colleagues on both sides of the aisle for their support i truly believe that this particular one was certainly the most important in helping new york rebuild and rebound i want to add that it did not cost our taxpayers one single dime it is an innovative way to get building and construction happening across this country so it is tremendously important to the economy it is an important bill and i am so pleased that it has been a bipartisan effort this body passed the bill it stalled in the senate but we do need to reauthorize it as swiftly and as quickly as possible i hope it is an example of how this body can work together on legislation that is critical to this country to rebuild and expand the jobs and our economy and to help strengthen our country in other ways so again i thank the leadership on both sides of the aisle for moving so swiftly to bring it to the floor and really to mr neugebauer who was the point person in many ways in the compromise legislation that moved forward i urge my colleagues to vote for it it is the right thing to do for america mr speaker i yield back the balance of my time
\vspace{8mm}
i yield myself such time as i may {\bf \color{red} consume} h r 1815 clarifies and updates several laws related to the management of federal land in eastern nevada this bill is cosponsored by the entire nevada delegation and i recognize its passage is important to the people of eastern nevada i want to thank the majority and the sponsor for working with the bureau of land management to address many of their concerns resolving those concerns and working with the blm turn this bill into a proposal we can support mr speaker i urge my colleagues to vote in support of this legislation i yield back the balance of my time\pagebreak

\section*{century}
h r 4084 the nuclear energy innovation capabilities act directs civilian nuclear energy research and development to contribute to american nuclear power i thank the energy subcommittee chairman randy weber and science committee ranking member eddie bernice johnson of texas for their leadership on this issue i also want to thank many bipartisan cosponsors of the bill which include science committee vice chairman frank lucas research and technology subcommittee chairwoman barbara comstock and subcommittee ranking member dan lipinski environment subcommittee chairman jim bridenstine oversight subcommittee chairman barry loudermilk space subcommittee chairman brian babin and full committee members dana rohrabacher ed perlmutter randy hultgren paul tonko bruce westerman steve knight bill posey and randy neugebauer i am encouraged by the strong bipartisan support for the subsequently introduced senate version of the nuclear energy innovation capabilities act which passed as an amendment to the energy policy modernization act by a vote of 87 4 on the senate floor in january advanced nuclear energy technology is the best opportunity to make reliable emission free electricity available throughout the modern and developing world america must maintain a strong nuclear technology sector in order to influence global nonproliferation standards this will help us prevent civilian nuclear energy technology from being misused for weapons development overseas h r 4084 harnesses the strengths of the department of energy doe national labs universities and the private sector it ensures that america s best and brightest minds advance this groundbreaking science and technology this legislation provides doe with the direction and certainty it needs to develop plans for long term research and infrastructure development within the office of nuclear energy h r 4084 authorizes doe to take advantage of the national labs supercomputers in order to accelerate research for advanced fission and fusion experimental reactors this program will leverage expertise from the private sector universities and national labs the bill provides a clear timeline for doe to complete a research reactor user facility within ten years this research reactor will enable proprietary and academic research to develop supercomputing models and also design next generation nuclear energy technology finally h r 4084 creates a reliable mechanism for the private sector to partner with doe labs to build fission and fusion prototype reactors at doe sites nuclear power has been a proven source of safe and emission free electricity for over half a {\bf \color{red} century} now america s strategic investments in advanced nuclear reactor technology can play a more meaningful role to reduce global emissions unfortunately the ability to move innovative technology to the market has been stalled by government red tape by working around these bureaucratic barriers h r 4084 will spur american competitiveness and keep us on the forefront of nuclear energy technology this legislation enables our talented engineers in the private sector academia and at the national labs to develop the next generation of nuclear technology here in the united states nuclear energy can be a clean cheap answer to an energy independent pro growth secure future i thank chairman weber and ranking member johnson of texas for their work on this bill and encourage my colleagues to support it
\vspace{8mm}
i rise today in support of h r 2064 the improving access to capital for emerging growth companies act i was pleased to introduce this legislation with my colleague congressman john delaney of maryland this legislation builds upon the success of the original bipartisan jobs act which i worked on that created a new category of stock offering for emerging growth companies which have proven to be a major new source of job creation for the 21st {\bf \color{red} century} job creation is the number one reason to support this legislation as companies are able to expand and go public they are able to hire more employees and to ultimately invest more in our economy our bill makes important changes to the registration process to ensure that these companies have the most efficient streamlined access to the market shortening the 21 day filing period to 15 days would save companies exposure to some market volatility before public launch the purpose of the 21 day period is to allow the information about the egc ipo to disseminate to the public before purchase orders are taken on the egc s stock but with today s technology the current 21 day quiet period is unnecessarily long the shortened time period would allow the benefit of clearer visibility in market conditions and would save companies from having to update financials and other disclosure before public launch additionally the bill calls for a grace period of the jobs act protections to an issuer who loses egc status mid ipo process under current law if a company exceeds the egc status criteria during the ipo process it no longer qualifies for the designation this discourages a borderline egc which may be considering going public from making an offering the grace period would allow an issuer who qualifies as an egc at the time of filing its confidential registration statement for review to continue to be treated as an egc through the date on which it completes its initial public offering or 1 year has passed whichever comes first finally the bill would permit egcs to avoid incurring the significant expense and effort of preparing and having audited financials and related disclosures for past periods that will not be included in the prospectus to investors this legislation was reported out of committee unanimously and i urge my colleagues on both sides of the aisle to support the passage of h r 2064 today this is a simple adjustment to reduce the burdens placed on smaller companies that are trying to access the market grow their businesses and hire more employees now more than ever as members of congress we need to be focused on ways to facilitate job creation this bill is an important step in that direction
\vspace{8mm}
i move to suspend the rules and agree to the resolution h res 939 expressing the sense of the house of representatives that access to digital communications tools and connectivity is necessary to prepare youth in the united states to compete in the 21st {\bf \color{red} century} economy\pagebreak

\section*{consent}
i ask unanimous {\bf \color{red} consent} that all members may have 5 legislative days in which to revise and extend their remarks and insert extraneous materials into the record on h r 5910
\vspace{8mm}
i ask unanimous {\bf \color{red} consent} that all members may have 5 legislative days in which to revise and extend their remarks and include extraneous material on the bill under consideration
\vspace{8mm}
i yield to the gentlelady from florida ms castor for the purpose of a unanimous {\bf \color{red} consent} request\pagebreak

\section*{underlying}
i thank the gentleman for yielding madam speaker i rise today in support of the rule and the {\bf \color{red} underlying} bill h r 5544 the minnesota education investment and employment act this bill will support the teachers and schoolchildren in the state of minnesota create well paying jobs in northern minnesota and make the boundary waters canoe area for the first time in its existence whole we have to have a bit of context here when minnesota became a state in 1858 sections 16 and 36 of every township in minnesota were set aside in trust for the benefit of schools the state could use lease or sell the land to raise money for education in the beginning the state leaders decided to sell the more valuable parcels of the school trust lands but around the turn of the century they realized they needed a more sustainable plan and began putting the school trust lands to productive use for timber and mining this has been the goal of the state for over 100 years and it has produced dividends for generations for our school kids as dfl state representative denise dittrich has so ably educated me on these lands are not so much owned by the state as held in trust by the state and owned by the schoolchildren of minnesota it is the responsibility of the school trust fund trustees to maximize the return on these lands for the benefit of this fund this is a critical point this is part of the minnesota constitution but in the 1970s the federal government created the boundary waters canoe area wilderness these lands within the boundary waters cannot be logged leased or mined in order to preserve the unique wilderness character of this pristine land thousands of visitors from around the country come to enjoy this beautiful area but as a result of its creation minnesota and its students have been faced with an 86 000 acre problem for over 30 years eighty six thousand acres of state owned school trust lands have been locked within the borders of the boundary waters canoe area unable to produce critical funding for minnesota public education it is imperative we resolve this longstanding problem our goal is to preserve and protect the boundary waters and allow state owned school trust lands to raise revenue for minnesota education unfortunately minnesota school kids have been cheated out of public education funding for over 34 years now in the past there have been a number of working groups studies and resolutions finally after years of inaction stalling and dilatory tactics by special interest groups republicans and democrats together in minnesota said enough is enough it s been referred to as mr cravaack s bill that is not in fact the case this is minnesota s bill on march 22 of this year an overwhelming majority of democrats and republicans from the state senate passed senate file 1750 on a vote of 53 11 on april 3 the house followed suit passing a bipartisan bill 90 41 on april 27 our democratic governor governor mark dayton signed the bill into law h r 5544 executes a bipartisan state plan that governor dayton signed into law earlier this year h r 5544 would exchange state owned school trust lands trapped in the boundary waters canoe area wilderness to the federal government in exchange for federal government owned land outside the boundary waters canoe area wilderness this bill includes important provisions that would ensure minnesotans can maintain their existing hunting and fishing rights within the boundary waters in addition the bill exempts the land exchange process from nepa the land exchange itself would have no environmental impact on any future development and would still be subject to strict state and federal regulations intuitively a land swap is merely a redrawing of maps and has no environmental impact in and of itself the mentioned activities mining and logging do in fact have environmental impact and would be subject to the full federal and state review not one environmental protection is lost in the execution of this bill i want to be very transparent here one of the hopes of my constituents is to have a bill to create good paying jobs in the timber and mining industries the lands listed in s f 1750 are rich in natural resources many of them lie in portions of the superior national forest that are already being successfully mined for iron ore and harvested for timber it s a working and managed forest these activities employ thousands of workers and support tens of thousands of other ancillary jobs in the region northern minnesotans want these and need these opportunities and every american benefits from the steel and lumber that goes into our cars and into our homes while i generally support the aims of nepa the state of minnesota has some of the strictest environmental standards in the country and a track record of successful regulation of mining and logging on the other hand obstructionist special interest groups have a track record of abusing the nepa process to sue and delay i do not want these groups to continue to delay this land exchange preventing minnesota schools from receiving the funding that they need and quite frankly they deserve the state of minnesota cannot afford to be sued by environmental groups for years some of those arguing for nepa are in fact arguing that defending lawsuits is an appropriate use of the taxpayer dollars and that it s okay to transfer wealth from state coffers to special interest groups interesting to note many of these special groups aren t even from minnesota make no mistake this will be passed and a bipartisan land exchange is going to get done i will not allow special interest groups acting in bad faith to abuse the nepa process and use frivolous lawsuits to block and derail a land exchange if i could trust special interest groups to act in good faith and if i could trust the federal bureaucracy to act promptly i would include nepa in this legislation the teachers and schoolkids in minnesota can t wait years if not decades currently some of the schools in minnesota have classrooms with over 40 kids and some school districts like mine in north branch have been reduced to a 4 day school week i ask is that progress this legislation will generate a lot of funding for our schools and create good paying jobs importantly the minnesota education investment employment act will not eliminate a single acre of boundary waters land in fact it would include wilderness acres to the existing boundary waters canoe area wilderness boundaries while giving minnesota s children land that rightfully and constitutionally belongs to them i urge my colleagues to support this rule and the {\bf \color{red} underlying} bill
\vspace{8mm}
i yield myself such time as i may consume mr speaker house resolution 481 and the {\bf \color{red} underlying} bill h r 3180 ensure the men and women of our intelligence community have the tools and the resources they need to continue the vital role they play in helping us address the threats facing our nation i do want to point out mr speaker my colleague is accusing the republicans of being brazen and reckless because of our same day rulemaking authority i would only note that in the 111th congress when they were in the majority they enacted this process 26 times in the 110th congress 17 times it is an important ability for us to have when we know we might need to move quickly on something as a member of the rules committee i am committed to ensuring we do everything possible to make sure that we are able to bring bills to this floor that carry out the kind of changes and improvements that the people of this nation sent us all here to undertake i was really disappointed mr speaker earlier this week when this bill was defeated under suspension of the rules there are many things that are partisan issues in this body and it is too bad when the minority uses the intelligence authorization bill as part of a political stunt to make what should be a bipartisan process and a bipartisan committee appear partisan the bill h r 3108 received unanimous support in committee and i certainly hope today mr speaker that the minority members of that committee and frankly all on the minority side who understand the importance of the intelligence community in keeping our nation safe will put aside the partisanship will put aside the games that the american people are so tired of and will join me in supporting a good effective and important bill that frankly the minority members in the intelligence committee worked very hard to help craft mr speaker i urge adoption of both the rule and h r 3180 mr speaker i yield back the balance of my time and i move the previous question on the resolution the previous question was ordered
\vspace{8mm}
today mr speaker the republican house takes an important step in restoring the trust of the american people in their elected representatives and in restoring the rule of law in our nation two of the most important principles {\bf \color{red} underlying} our entire system of government are trust and the rule of law the american people in the election last november decisively rejected the aggressive liberal agenda of this president and of the democrats in congress they elected this republican majority to stop the president from doing further damage to our system of laws and further damage to our constitution the american people elected us to preserve and protect and defend the constitution of the united states but that work begins with trust we today are doing what the voters of america asked us to do in enforcing our laws on the border to ensure that our laws are respected to ensure that our immigration law is fair and that it treats everyone equally as the constitution requires we are keeping our word to the american people to do precisely what we said we would do and that is to overturn these illegal executive memos that are attempting to ignore what the law says the president must do not even king george iii had the authority to waive a law enacted by the parliament mr speaker once we have begun this path today of restoring that bond of trust we will restore the rule of law in america because without the law there is no liberty in fact the first design on one of the first coins ever minted in the republic of mexico a coin which i have here with me shows the liberty cap liberty and law there is no liberty without law enforcement and the house today is doing what the american people hired us to do to restore their trust and to restore the rule of law this is a law enforcement issue border security and immigration these are matters of law enforcement we trust the good hearts and the good sense of the officers in the field to do the right thing for the right reasons which is to enforce our laws fairly and equally because the people on the rio grande understand better than anyone else that if the law is not enforced there cannot be safe streets and that you cannot have good schools and a strong economy without law enforcement we in texas understand better than anyone else that this debate is far larger than it just being about immigration or border security it is far larger than just these individual issues we will debate today today we in the republican house are honoring the will of the american people we will keep our word we will make sure that the laws of the united states are enforced equally and fairly for all above all we will preserve and protect the constitution and the america that we know and love that was the message of the election last november\pagebreak

\section*{vote}
i yield myself such time as i may consume madam speaker i rise in very strong support of this measure and i want to especially thank the gentleman from north carolina mr mark meadows along with mr ted deutch of florida and ranking member eliot engel of new york for their bipartisan leadership on this critically important issue last july the house passed legislation by a {\bf \color{red} vote} of 404 0 this was the bill that was passed by that measure with a few tweaks but 404 0 unfortunately the other body the senate failed to take it up the threat posed by hezbollah and other iranian proxies has only expanded since then and now hezbollah is ascendant in the region consider now hezbollah s arsenal aimed at israel that arsenal has exploded i was in haifa in 2006 as hezbollah s rockets rained down on that city targeting civilian neighborhoods those iranian and syrian made rockets were slamming into people s homes and they were being targeted and the hospital also was being targeted every rocket contained 90 000 ball bearings the only intent was mass killing and maiming in the rambam trauma hospital i talked to many of the victims there were 600 victims of these rockets in there and that was nearly 10 years ago at that time hezbollah started that effort with about 15 000 rockets at their disposal and they fired close to 5 000 at civilian targets that was their work hezbollah has expanded its arsenal in size and in sophistication by the way it has been done at the behest of iran they have given these new rockets with longer range to hezbollah now they have an arsenal the estimate is some 100 000 unguided rockets it has also expanded its arsenal to include the sophisticated antiship and antiaircraft missiles and ground to ground rockets hezbollah has been able to expand both its arsenal and activities with iranian backing and its long established worldwide network of members and supporters and sympathizers to provide this terrorist group financial and logistical and military and other types of support to cut the international support and reach of hezbollah to deny it the funds needed for its terrorist activities we must effectively target its financial network that is the goal of the hezbollah international financing prevention act of 2015 this bill builds on the existing sanctions regime by placing hezbollah s sources of financing under additional scrutiny particularly those resources outside of lebanon given that many lebanese banks have stepped up their game to prevent money laundering in addition to targeting the terrorist organization s diverse financial networks the legislation also requires the u s government to focus on hezbollah s global logistics network and its transnational organized criminal enterprises including its vast drug smuggling operations the goal is to improve coordination and cooperation with allies and other responsible countries in confronting the increasing threat posed by hezbollah and i strongly urge my colleagues to support this critical measure madam speaker i reserve the balance of my time
\vspace{8mm}
madam chair this amendment would increase the defense health program account by 10 million in order to fund a cure for gulf war illness currently there is no cure for gulf war illness and it affects over a third of the veterans who served in the first gulf war this amendment is identical to an amendment offered last year that passed this body by a voice {\bf \color{red} vote} and according to the congressional budget office this amendment actually will reduce total outlays by 1 million veterans of the first gulf war suffer from persistent symptoms including chronic headaches widespread pain cognitive difficulties debilitating fatigue gastrointestinal problems respiratory symptoms and other abnormalities that are not explained by traditional medicine or psychiatric diagnoses research shows that as veterans from the first gulf war age they are twice as likely to develop lou gehrig s disease as their nondeployed peers there also may be connections to multiple sclerosis and parkinson s disease sadly there are no known treatments for the lifelong pain and affliction that these veterans must endure through this disease for decades the veterans administration has downplayed any neurological basis for this disease but recent research just this year has shown unequivocally that this disease is biological in nature the time has come to right the wrong that our servicemen and women have had to live with for over 20 years in this department of defense appropriations bill we allocate more money for breast cancer orthopedic and prostate cancer research than we do for finding a cure for gulf war illness equivalent funds are appropriated for ovarian cancer research personally i think if we are going to spend money on medical research within the department of defense the department must adequately fund research on those diseases that originate in war and wholly affect our servicemen and women over a quarter of a million veterans display symptoms of this disease and the time has come to find and fund a cure for it the offset for my amendment today comes from the 32 million operation and maintenance defense wide account and that account is funded 500 million above the amount in last year s dod appropriations bill congress has responsibility to ensure that the gulf war veterans who put it all on the line and are paying for that with a lifetime of pain are not left behind i urge my colleagues including my esteemed colleague from florida to support this amendment and help to find a cure for gulf war illness i reserve the balance of my time
\vspace{8mm}
i rise today as a vice chair of the house gun violence prevention task force and in solidarity with the majority of americans who are demanding that congress take meaningful action to prevent gun violence we all know the statistics whether it is through mass shootings that make the headlines or the unseen violence that happens daily on our streets gun violence takes the lives of more than 30 000 of our nation s citizens each year a number that far exceeds other industrial countries now all these countries have their share of violent extremists and madmen but only our country gives easy access to weapons of mass killing and that makes all the difference for america rather than seeking out commonsense solutions to address this crisis the republican majority continues to cower to the gun lobby and the firearms manufacturers now they plead the second amendment but constitutional law 101 would tell us that all of our rights including the precious freedoms of religion and speech must be balanced to protect innocent third parties and to protect the safety of the wider community one commonsense measure we should all agree on is background checks to keep guns out of the hands of criminals domestic abusers and the dangerously mentally ill you can t shout fire in a crowded theater because of your freedom of speech and neither should you be able to buy a weapon if you have a history of violence and criminality in fact almost 90 percent of americans including the majority of gun owners support universal background checks for all gun purchases the problem is that our present background check system is rife with loopholes background checks are not required for private sales at gun shows they are also not required for internet sales bipartisan legislation has been introduced by representatives peter king and mike thompson that would finally close this egregious loophole it is an entirely sensible reform that would have a measurable impact on the safety of our schools homes and neighborhoods without preventing law abiding citizens from using guns for self defense or recreational purposes despite attracting 186 cosponsors including several republicans the background check legislation has never been brought to the floor or even received a hearing in committee it has been languishing for more than 15 months meanwhile the shootings and the suicides and the massacres continue to accumulate my colleagues we must do better our fellow citizens are totally fed up both with the unspeakable killing and suffering and with a feckless congress that hasn t lifted a finger to prevent it now this week after intense public criticism and a historic protest by democrats on the house floor republicans seemed for a while to be willing to hold a {\bf \color{red} vote} on legislation they claim would prevent suspected terrorists from purchasing firearms after all nearly 2 500 individuals on the terrorist watch list have successfully purchased weapons in this country but rather than embrace existing bipartisan legislation to actually fix the problem republicans put forth a woefully inadequate proposal that would require law enforcement and courts to grapple with unworkable processes unreachable standards to be completed in an unreasonably short period of time their bill would allow suspected terrorists to receive firearms by default after only 3 days if the court is unable to work through a complicated process that is the same flaw that allowed the white supremacist charleston shooter to obtain the weapon that he used to murder nine people at emanuel ame church in other words the bill is totally inadequate now under pressure from their most extreme members republican leaders refuse to even put this bill on the floor what should be on the floor is bipartisan legislation h r 1076 that would permit the attorney general to block gun sales to suspected terrorists this legislation based on a proposal from the bush justice department would still allow individuals to challenge the government in court to restore their gun ownership rights we don t have to choose between protecting our communities and respecting due process and so mr speaker we ask our colleagues how much longer must we wait how many more people have to die to move us to act how many more american towns and cities must be added to the constantly growing list of places like orlando and columbine and aurora and charleston and newtown moments of silence aren t enough thoughts and prayers are not enough in fact the scriptures teach us that such pieties give grave offense when they mask a refusal to do what we know is right we need action i call on my colleagues to bring these commonsense proposals to the floor for a {\bf \color{red} vote}\pagebreak

\section*{chair}
i ask unanimous consent that it may be in order at any time on tuesday march 3 2015 for the speaker to declare a recess subject to the call of the {\bf \color{red} chair} for the purpose of receiving in joint meeting his excellency binyamin netanyahu prime minister of israel
\vspace{8mm}
mr chairman i yield myself such time as i may consume good morning mr conyers it is good to see you six long years into the obama administration and notwithstanding some fleeting recent signs jobs have yet to recover from the recession wages also have not recovered and the rate of new business startups has not recovered as well instead permanent exits from the labor force are at historical levels real wages have fallen dependency on government assistance has increased our economy is failing to give enough hardworking americans the confidence they need to start new small businesses and create new jobs at the root of our problem are more than anything else the endless drain to washington of hard earned income that working people and small businesses need to turn things around in their homes and communities and washington s endless placement of regulatory roadblocks in the path of opportunity and growth that regulatory burden hits small businesses especially hard small businesses generate 63 percent of net new private sector jobs and employ nearly half of america s private sector workers yet they have to pay significantly more to comply with federal regulations than do larger employers poll after poll has demonstrated that the level of federal regulations coming from washington is at the top of the list of obstacles faced by america s small businesses our top job creators this is not fair and it is exactly the wrong burden to place on small businesses as this nation struggles to produce a true jobs and wages recovery congress can and should act to free small businesses of the burdens and waste associated with excessive federal regulations so that more jobs will be available to americans trying to make a better life for themselves and their families that is why prompt passage of the small business regulatory flexibility improvements act is so important this legislation will for the first time in nearly 20 years overhaul the laws that govern how federal regulators should consider and minimize the adverse impacts of new regulations on small businesses primarily the bill reinforces the regulatory flexibility act of 1980 and the small business regulatory enforcement fairness act of 1996 it only requires agencies to do what current law tries to achieve and what common sense dictates should be done however current law is beset by loopholes and those loopholes must be closed that is what the small business regulatory flexibility improvements act at long last does for example the bill mandates that all agencies not just the current few work with small business review panels early in the rulemaking process for major rules before agencies become entrenched in their proposed paths to help small businesses better and more effectively point out to agencies what is the best path the bill also requires agencies to assess not just the direct effects of new regulation on small businesses but also indirect effects which often can be substantial the bill also for the first time authorizes the small business administration s chief counsel for advocacy to be the one consistent authority on regulatory flexibility requirements the law imposes on all agencies this will at long last curb the agencies tendencies to interpret the law to suit their own individual whims and will force agencies to focus on the common needs of small business the minute this bill becomes law what will start to happen small businesses will have a real chance to be heard before agencies effectively make up their minds agencies will have better information upon which to tailor their regulations to reduce unnecessary burdens on small businesses agencies will have fewer opportunities to escape requirements to hear those businesses and gather that better information and small businesses will be freer than they have been in decades to devote their resources to what they do best create the new jobs products and services that can drive the economy forward to true and lasting recovery the small business regulatory flexibility improvements act recognizes that economic growth ultimately depends on job creators not regulators it represents a critical means to convert the recognition into reality mr {\bf \color{red} chair} i reserve the balance of my time u s congress congressional budget office washington dc february 3 2015 hon bob goodlatte chairman committee on the judiciary house of representatives washington dc dear mr chairman the congressional budget office has prepared the enclosed cost estimate for h r 527 the small business regulatory flexibility improvements act of 2015 if you wish further details on this estimate we will be pleased to provide them the cbo staff contact is matthew pickford sincerely douglas w elmendorf director congressional budget office cost estimate h r 527 small business regulatory flexibility improvements act of 2015 summary h r 527 would amend the regulatory flexibility act rfa to expand the number of rules covered by the rfa and to require agencies to perform additional analysis of regulations that affect small businesses the legislation also would provide new authorities to the small business administration s sba s office of advocacy to intervene and provide support for agency rulemaking finally h r 527 would require the government accountability office gao to report on the implementation of the legislation cbo estimates that implementing h r 527 would cost 55 million over the 2015 2020 period assuming appropriation of the necessary funds enacting the bill could affect direct spending by agencies not funded through annual appropriations therefore pay as you go procedures apply cbo estimates however that any net increase in spending by those agencies would not be significant enacting h r 527 would not affect revenues h r 527 contains no intergovernmental mandates as defined in the unfunded mandates reform act umra and would not affect the budgets of state local or tribal governments if any federal agencies increase their mandatory fees to offset the costs of implementing the additional analysis required by the bill h r 527 would increase the cost of an existing mandate on private entities to pay those fees cbo expects that if such mandatory fees are increased as a result of the bill the additional cost of the mandate in any one year would fall well below the annual threshold established in umra for private sector mandates 154 million in 2015 adjusted annually for inflation estimated cost to the federal government the estimated budgetary effect of h r 527 is shown in the following table the costs of this legislation fall within budget functions 370 commerce and housing credit 800 general government and all budget functions that include funding for agencies that issue regulations affecting small businesses by fiscal year in millions of dollars
\vspace{8mm}
mr {\bf \color{red} chair} i yield back the balance of my time\pagebreak

\section*{special}
i rise today in recognition of bioenergy day a day we celebrate natural renewable energy in america on october 18 organizations across america will mark this {\bf \color{red} special} day by opening their doors to the public and highlighting how bioenergy is fueling america forest by products are a primary source of bioenergy making my home state of arkansas a leading producer in this field across the country bioenergy keeps the lights on and so much more bioenergy produces just under 6 percent of the nation s total energy supply and provides full time jobs for tens of thousands of americans with more plants coming online in the near future we need to do more research to find economical ways to harness renewable energy in our abundant biomass that we all too often continue to see going up in flames and wildfires as we approach october 18 i encourage all americans to learn more about bioenergy forest by products and the environmental benefits derived from our natural resources
\vspace{8mm}
i rise today to celebrate everson s hardware in waconia minnesota for 50 years of business success ron and mary ann everson bought the store back in 1966 when they were just a young couple with two growing children throughout the years everson s hardware has become a well respected and established part of the community and the everson family has realized their american dream eventually ron and mary ann passed the store along the way to tracy and deborah everson who continue to work behind the counter in this family store today small family operated businesses are what make minnesota so great they make our community {\bf \color{red} special} i want to thank the everson family for their lasting contribution to waconia congratulations and best of luck on the next 50 years
\vspace{8mm}
i rise today to praise a school in my district that is going the extra mile to ensure that our heroes are given the resources and opportunities they deserve on april 22 paul smith s college in my district will hold the grand opening of their veterans resource center a project made possible through grants from the holder family foundation and the fred l emerson foundation this center will provide vets on campus with an important recreational resource and place to study as well as house the paul smith s college veteran s club paul smith s has a proud tradition of service with our armed forces dating back to when the campus was first built during world war ii and was used as a u s army signal corps training center since then the school has upheld their commitment to our servicemembers working with several organizations to ease the transition from service to academia for our veterans mr speaker i am pleased to speak on the house floor today to recognize the {\bf \color{red} special} work being done for our heroes by paul smith s college\pagebreak

\section*{previous}
if we defeat the {\bf \color{red} previous} question i will offer an amendment to the rule to bring up h r 3440 the dream act this bipartisan bicameral legislation would help thousands of young people who are americans in every way except on paper mr speaker i ask unanimous consent to insert the text of my amendment in the record along with extraneous material immediately prior to the vote on the {\bf \color{red} previous} question
\vspace{8mm}
i yield myself such time as i may consume mr speaker one of the reasons why many of us are opposed to h r 3905 is because we see it as a corporate giveaway that puts treasured public lands in the hands of a chilean mining conglomerate so much of what comes out of this congress is about rewarding those who are well connected and well off rewarding corporations at the expense of average citizens i will go back to our {\bf \color{red} previous} question which would force a vote everything would still move forward but it would force a vote on a bill that was introduced by ms eshoo that would require presidents and presidential nominees to release their tax returns it is not just the connections between the trump family and this mining company that we have concerns about it is the connections between this president and his administration and the tax bill that is being proposed that we know would raise taxes on millions and millions of middle class families and basically give a big tax cut to the wealthiest individuals and to corporate special interests and to corporations we think that is all backwards but i think the american people deserve to know who benefits and who doesn t again for the life of me i don t understand why so many of my republican friends have circled the wagons in opposition to transparency in opposition to letting the american people know where this president s conflicts of interests are and basically protect what i think may very well be multiple conflicts of interest and maybe conflicts of interest that lead directly into a collision course with corruption this is a big deal all this legislation that we are talking about here today there is a good piece of legislation the brownfields legislation a bad piece that is this mining bill this still goes forward but vote with us to defeat the {\bf \color{red} previous} question so that we can bring up this other bill now my republican colleague from wyoming may say well that is not what we are talking about here today the democrats are just trying to muddy up the discussion the reason why we have to resort to a procedural motion to bring up this bill to force the president and presidential nominees to release their tax returns is because the rules committee shuts everything down that this leadership doesn t want to see come to the floor we can t bring this bill to the floor to require the president to release his tax returns under regular order and our normal process they won t let us we can t offer it as an amendment they won t let us this is the only way we can do it i would urge my republican friends to stop defending the indefensible here this thing would apply not just to donald trump it would apply to every president we have never had to do this before because every other president has released their tax returns this president for some reason doesn t think it is anybody s business given the nature of the legislation coming out of this house of representatives i think the american people need to know and have a right to know mr speaker i reserve the balance of my time
\vspace{8mm}
i yield back the balance of my time and i move the {\bf \color{red} previous} question on the resolution\pagebreak

\section*{legislative}
mr chair i wish the gentlewoman had made that same speech when we were discussing defense the biggest spending bill we have but she didn t offer this amendment at all i happen to come from a state where the legislators didn t have enough guts to raise taxes so the people went out and did it because they want their government to run wisely and smartly and they knew they didn t have enough money to do it look we are cutting this budget yet the senate which we don t vote on their bit is increasing their budget by 12 percent they are going to be able to give cost of living adjustments to every one of their members nobody sitting in this room who works for us is going to get a cost of living adjustment because of cuts like this this is ridiculous we are penalizing our whole house not the senate this is not a smart way to make {\bf \color{red} legislative} business
\vspace{8mm}
i ask unanimous consent that all members may have 5 {\bf \color{red} legislative} days to revise and extend their remarks
\vspace{8mm}
i ask unanimous consent that all members have 5 {\bf \color{red} legislative} days within which to revise and extend their remarks and submit extraneous materials for the record on h r 1726 as amended currently under consideration\pagebreak

\section*{21st}
i probably will not use all of the 5 minutes but i wanted to be here today mr speaker to support h r 3547 the space launch liability indemnification extension act of which i am an original cosponsor i want to thank both our chairmen the chairman of our subcommittee mr palazzo and chairman smith and of course our ranking democrat on the committee ms johnson because we would not have been able to get to this point if we hadn t been able collectively across the aisle to work on a 1 year extension that would be provided for in commercial space launch act amendments of 1988 that established the government private risk sharing regime for third party liability should a launch accident occur the effects that involve the public and property on the ground in this indemnification provision would cover such losses it turns out that commercial space launch capacity in the industry is really at a critical point in our nation s development of our space infrastructure both the federal and commercial customers rely on commercial space launch the industry for safe reliable and effective service and delivering payloads in orbit and providing related space transportation services just recently in september of this year a commercial space launch provider successfully lofted a cargo capsule into space to carry supplies to the international space station this is exactly what we have in mind when we talk about integrating our commercial launch capacity with what we do already at nasa in terms of our scientific endeavors mr speaker commercial space transportation services have really always been carried out in partnership with the united states government through the use of federal launch ranges and services for example and through the government risk sharing regime for protecting the uninvolved public and property should an accident occur so it seems quite fitting that we have reached this point today unfortunately the reason that we are only able to do a 1 year extension and can agree on that is because there are also some other things that we need to figure out for the future with respect to the involvement of the commercial industry it is my hope that over the course of this 1 year we will use that time wisely here in the congress to have the kind of oversight hearings that we need to bring in the faa so that we can make sure that we are venturing in this direction in the right kind of way that really takes into consideration what we are doing in the {\bf \color{red} 21st} century new entrants are delivering spacecraft to orbit commercial resupply services to the international space station and companies are working toward providing commercial human spaceflight on both reusable suborbital vehicles and orbital human spaceflight systems in fact although i have been admittedly a skeptic i am excited about the potential of the industry and i want it to succeed just last year in a hearing on launch indemnification before the committee s space subcommittee of which i am the ranking member a senior official representing the aerospace industries association characterized the continuation of u s space launch indemnification as providing substantial upside potential to enable new markets create jobs and assure u s space technology leadership for the {\bf \color{red} 21st} century it is easy to see how that upside is both national and local in scope the launch capability at nearby wallace air facility on the eastern shore is becoming a critical link to resupplying the international space station commercial space companies make investments in our economy and create jobs all across the country specifically in my home state of maryland companies like lockheed martin orbital and northrop grumman employ thousands of people in my district alone creating high tech jobs high skilled jobs in the local community atk is a leading aerospace provider and has its main headquarters right up in beltsville maryland not very far from here mr speaker i want to ensure that our legislation and policies regarding commercial space transportation reflect the changing industry changes and activities that may not have been contemplated when the liability indemnification regime was first established this 1 year extension provides congress the opportunity to consider any potential changes that might be needed to ensure the continued safety of the public mr speaker i urge our colleagues to join us today in supporting h r 3547
\vspace{8mm}
mr chair before i start i just want to thank the gentleman from idaho the chairman of the subcommittee on energy and water development and related agencies appropriations for all his exemplary work on this and allowing the open process to actually work i know that my colleagues that have offered amendments appreciate the time that they have been able to do that and i want to thank him for that mr chair i rise today to offer an amendment to the energy and water appropriations bill this amendment takes dollars out of the bureaucracy in washington d c and puts it to work for the american people helping ports and harbors like the charleston port in my home state of south carolina do the important work necessary to begin the deepening of those harbors last month i had the pleasure of visiting the panama canal when i led a house delegation to the summit of the americas the lock and dam system in that canal is being upgraded and it was very interesting to see the work that they are doing there once that work is complete larger ships will be able to come through the canal and deliver goods to and from atlantic and gulf ports along the eastern seaboard this will be one of the key economic drivers in the {\bf \color{red} 21st} century if america is going to compete on the global stage we need to be ready for this transformation my amendment seeks to speed that readiness helping to transform critical ports like charleston s to the depth that will allow these bigger ships to navigate those harbors more often this amendment is about this house setting our government s spending priorities just like every family does at home we are rapidly approaching a 20 trillion debt and we have a moral responsibility to use every tax dollar wisely i am grateful that my colleagues on the committee on appropriations were able to negotiate this amendment to increase funding for vital infrastructure projects like the port of charleston and pay for it by forcing bureaucratic agencies to operate more efficiently i urge the passage of the amendment i yield back the balance of my time
\vspace{8mm}
i thank the gentlewoman from new jersey for yielding and i stand with her and what she has just said mr speaker sometimes we forget how privileged we are the members of congress who have a chance to stand in this hallowed chamber we are the representatives of the people we get elected to speak for the american people we get elected to act on behalf of the american people very few americans throughout the history of our country have had an opportunity to stand right here where we are today and say that we actually can get things done not just for the american people for the people of the world because there has never been a democracy like the united states of america there has never been a country that has had an opportunity to do so much for so many and there has never been a democracy that has a chance to prove to the world that we know how to get this done and do it right mr speaker as we stand here in this chamber we have to admit we have to be prepared on behalf of the american people to stand up to step up to do what is right and to do what the american people expect us to do now they know we have to speak for them but they don t want us just to talk the time to just talk on so many issues has come and gone mr speaker i think the american public would agree that the time to just talk about what to do about the confederate battle flag has come and gone the time to just talk about what to do about the confederate battle flag came 150 years ago when the chance to heal was upon us as president lincoln said in his second inaugural address with malice toward none with charity for all with firmness in the right as god gives us to see the right let us strive on to finish the work we are in to bind up the nation s wounds if we needed to talk abraham lincoln said it all lincoln wanted us to act to move to get things done for the american people the time to talk came after one after another black church was suspiciously burned down throughout this country and we knew something was going on that was the time to talk about what we needed to do the time to talk was before a man driven by hate and animosity on june 17 entered mother emanuel ame church in charleston south carolina to carry out a vicious plan to start a race war because we have seen these signs of danger growing for the disregard for life that would have been a time to talk and heal before that man crazed with hate walked into mother emanuel church but mr speaker after nine innocent god loving god fearing americans were taken from their families from their church where they were praying from their country the time to just talk is over it is time for us to step up it is time for us to stand up because that is why we get elected to do what the people expect us on their behalf to do 320 million americans cannot get up and say it is time to remove the confederate battle flag from any grounds where we reflect the governance of a democracy they encharge us to do that and the time to talk has ended when we see on the floor of the house last night an opportunity for the congress to register itself and say we hear you america you want us to act and you want us to take down that confederate battle flag in whatever symbolic way we can including selling that symbol here in the capitol we had an opportunity in fact we had an opportunity that was golden because it seemed like we had a bipartisan vote to do exactly that but in the dead of night something happened some people decided to hide behind the dark cloud and change what we had just done when we take to the floor here we may only be talking but as my colleague from new jersey said we are going to do much more because the time to talk has just ended it is time to act it is time to step up we all have an opportunity we all have an obligation to stand up tomorrow morning at 10 the confederate battle flag will finally come down from above the south carolina capitol once and for all mr speaker the confederate battle flag has no place but a museum in the {\bf \color{red} 21st} century let us all together those of us privileged to be in this chamber along with our fellow americans forge a path forward as a nation that celebrates our bright future not our dark past it is time to take the confederate battle flag down it is time for us to step up it is not a time to hide behind procedural motions behind votes in the dead of night and it certainly is not time for us to assemble a bipartisan group of members to talk about what we need to do about the confederate battle flag it is time to do the work of the people and they want us to act there should be no doubt about it the american people are speaking very forcefully don t just talk act mr speaker i say with great pride having served in this chamber for many years i believe the people s representatives in the people s house are getting ready to act and no act during the dead of night no effort to derail this effort will succeed because the people have spoken and spoken in the words of the nine people who are no longer with us we do it with grace but we will do it with power because we understand this is not a time to just talk it is a time to act and we will act i thank the gentlewoman for yielding\pagebreak

\section*{floor}
today i am proud to speak in support of h r 5405 and i do want to thank representative fitzpatrick from pennsylvania for his important work on this bill among other things this bill will help encourage capital formation at small and emerging businesses these tools helps businesses expand their operation and most importantly hire more workers i am especially pleased that the bill includes my own legislation the encouraging employee ownership act of 2014 or eeoa this bipartisan provision would make it easier for companies in illinois and nationwide to let hardworking employees own a stake in the business they are a part of i have learned firsthand from my constituents in the 14th congressional district about the many benefits of employee ownership when you walk into scot forge an entirely employee owned manufacturer in my district there is a noticeable difference in the energy of the employees from upper management on down to the shop {\bf \color{red} floor} when employees have a stake in the company they work for their sense of ownership over details large and small makes a real difference to their bottom line and more importantly to their quality of life the business in turn receives a large boost in productivity enabling them to expand their reach and invest in new technologies and equipment unfortunately some companies are shying away from offering employee ownership because of regulations that limit how much ownership they can safely offer sec rule 701 mandates various disclosures for privately held companies that sell more than 5 million worth of securities for employee compensation in 1999 the sec arbitrarily set this threshold at 5 million without a concrete explanation why for businesses who want to offer more stock to more employees this rule forces those businesses to make confidential disclosures that could greatly damage future innovations if they fell into the wrong hands the sec s original rulemaking acknowledged this and some voiced their concern that a disgruntled employee could use this confidential information to harm their former employer further it is costly to prepare these disclosures just so a business can offer the benefits of ownership to their employees my bill included in h r 5405 would address this problem as the chamber of commerce who supports this legislation has explained this legislation would help give employees of american businesses a greater chance to participate in the success of their company i want to thank representatives bachus fitzpatrick garrett hurt mulvaney ross and stivers for their support it is also worth noting that in good faith both sides agree to lower the threshold to 10 million instead of the 20 million the bill originally included i am glad we could iron out our differences and put forward a strong bill i want take thank my colleagues on the other side of the aisle including representative jared polis of colorado for his support and representative john delaney of maryland for his hard work on this bill the question remains do we want businesses to reserve employee ownership only for senior level executives because of concerns about costs or the dissemination of confidential information under my bill they will not be forced to make that decision because of this easier and safer method of offering ownership to more employees i encourage all my colleagues to support this legislation
\vspace{8mm}
if we re going to really be caring about the american worker going back to work we also need to be very cognizant of international competition you spoke earlier about the need for our workforce to be competitive which is the education process k 12 vocational education community colleges they re exceedingly important also important is that there be fairness in the international trade situation that we look not just for free trade but fair trade one of the things that we really must address is the threat of china s unfair trade practices the chinese currency is undervalued and as a result of that they have a 20 to 25 percent advantage you eliminate that and the american worker will be competitive we have one of the pieces of legislation in the make it in america package that the democrats are putting forward which is forcing china to end its currency manipulation when it ends its currency manipulation and allows the value of its currency to rise to appropriate parity we will be able to be competitive you can bet why the chinese don t want to do it they want that unfair trade advantage that s one of the pieces of legislation that we put forward when the democrats controlled congress a year and a half ago we pushed a bill out of here that would force sanctions on china if they continued their currency manipulation since the republicans have taken control of the house of representatives that legislation has died has never even come up for a vote on the {\bf \color{red} floor} it ought to come up for a vote we need fair trade practices we need to use our tax money to buy american made equipment and supplies we need to educate our workforces these are investments in the american middle class this is how we can restore the middle class of america health care is part of it also you talked earlier about health care and the availability of health care for working men and women we also need to make sure that those jobs are there the american automobile industry is instructive on this count it is instructive in that the u s government and the leadership of president obama actually allowed the american automotive industry to continue to even survive using the stimulus program the president stepped forward and said i will not allow the american automotive industry to die and he put our tax money behind general motors and chrysler those companies are now thriving and it s not just those companies it is the thousands upon thousands of manufacturers across this nation and others who supply all of the parts and all of the services think where we would be today if congress had not given the president the power and if this president did not have the courage to take up saving the american automobile industry presidential politics come here mr romney says he would not have done it okay president obama did it and the american automobile industry is strong and vibrant today and the american middle class is back to work mr tonko we must be about out of time
\vspace{8mm}
it s my privilege to be recognized by you here on the {\bf \color{red} floor} of the house of representatives and it s my privilege to follow the gentleman from texas as we close out this legislative week and a lot of the members are on their way to the airport or at the airport now going back to serve their constituents i ll be there myself and i trust mr gohmert will be too i wanted to come to the {\bf \color{red} floor} and talk about this country that we have this civilization that we have the foundations of our civilization and what s required to retain them and enhance them and move this country beyond the shining city on the hill and to a place beyond there onward and upward ronald reagan often described the shining city on the hill he described it as an america that is an america that was and an america that is we were always challenged by the dream but he didn t actually articulate that i recall something beyond this shining city but societies must progress and those that progress the most effectively and those that can be sustained the longest need to be built upon solid pillars the shining city on the hill standing true and strong on a granite ridge was built on a solid foundation and i argue that the foundation of it are the pillars of american exceptionalism and those pillars are listed in the bill of rights you add to that free enterprise capitalism judeo christian values the foundation of our culture which welcomes all religions and on top of that the dream that inspired legal immigrants to come to america and that dream embodied within the vision of the image of the statue of liberty that s the american dream that s the american country that we are and that s the foundation upon which we ve got to build our american future how did we get here what was the reason that these pieces came together how was it that our founding fathers came to a conclusion that we would have freedom of speech religion assembly the right to keep and bear arms freedom of the press that we would have property rights that we would have fourth amendment rights against unreasonable search and seizure that we would not have to face any kind of jury but a jury of our peers and that we would not suffer double jeopardy and that justice would be blind and every person would stand before the law to be treated equally the statue that we see of lady justice holding the scales of justice perfectly balanced is almost always shown to us blindfolded because justice is blind but justice is not a feeling justice is something that has to be delivered by the law these are pillars of american exceptionalism as are those rights that are not enumerated in the constitution that devolve respectively to the states or the people those enumerated powers that we have for congress or those delegated to the presidency the executive branch and the judicial branch of government all of this is laid out as foundations that have been although they ve been altered to some degree over the years we have adapted to those principles more often than we ve altered our constitutional principles because our founding fathers got it right where did that come from how could it happen that these founding fathers could come together on what was an obscure place on the planet and get these ideas so well articulated that they could be the foundation of the greatest nation the world has seen the strongest economy the world has seen the most dominant culture and civilization that the world has seen the furthest reach in our economy the furthest reach in our influence strategically how did this all happen and i would take you back mr speaker to think a little bit about the formation of i ll say modern history i take you back to mosaic law before the time of christ when moses who looks down upon us right now the only face that is looking directly at us from all of these faces of law providers in history moses looks down over this chamber in full face form and he s looking back here and he sees as we should see in god we trust our national motto how did that come together mr speaker it was when moses came down from the mount with the law god s law and the foundation of that law the way it was separated out through the tribes and the way the law and the way justice was delivered emerged out of mosaic law and appeared also in greek law and as the greeks masterful people as they were they were shaping the age of reason so we had mosaic law that informed the greek age of reason and the age of reason where i imagine that socrates and plato and aristotle and other philosophers sat around and challenged each other intellectually like gunslingers did in the west with guns they did it with their brains and young philosophers would go up to socrates and challenge him with their philosophy and socrates would take it apart because he was the top guy and he informed others but as they were proud and prideful of their ability to reason and the culture of greece at the time they had to infuse mosaic law to uphold their rationale and some of them as they voiced mosaic law were teased by other greeks that said well you got that from moses but my point in this is that as civilization was progressing mosaic law came down from the mount was handed to civilization it emerged through the greek civilization as the greeks were developing their age of reason and we re talking about the foundation of western civilization and almost concurrently with that roman law was emerging as well now i ll take you then to the time of christ christ taught us our values the very values of repentance and redemption that didn t exist in any form before then and that s his gift to us but i make this point in talking about the law and it is this think of mosaic law coming down being infused within the greeks transferred also to the romans roman law ruled over that part of the world where christ stood before the high priest caiaphas and if you remember mr speaker the high priest said to jesus did you really say those things did you really preach those things and jesus responded to the high priest as the jews were watching he said ask them they were there they can tell you that was mr speaker the assertion by jesus that he had a right to face his accusers that principle remains today in our law that we have a right to face our accusers and when he said ask them they were there they can tell you he s facing his accusers and demanding that they testify against him rather than make allegations behind his back and what happened when jesus said that they believed and the high priest believed that jesus answer was insolent and the guard struck jesus jesus said if i speak wrongly you must prove the wrong if i speak rightly why do you punish me he asserted his right to be innocent until proven guilty before a roman court those two principles remain today in our law a right to face your accuser innocent until proven guilty you face that jury of your peers as i said you need a quick and speedy trial they didn t have to wonder about that in those days it happened quickly and the punishment came quickly as well right or wrong this foundation of law was wrapped up in roman law and it was spread across western europe as the romans occupied areas like germany england as we know it today on into ireland and when the dark ages came when the visigoths sacked rome in 410 a d then we saw civilization itself tumble and crumble and we saw the heathens break down anything that represented the old culture anything that represented real civilization while that was going on they were tearing buildings into rubble they were burning anything that was written documents while that was going on the priests and let me say the descendants of the disciples of christ began to gather up any papers and documents they could get their hands on some went to rome to be secured and replicated by the monks and the scribes there a lot went to an island off of ireland where the monks and the scribes replicated those documents there that was the foundation of the relearning of a civilization a civilization that had been lived for centuries having lost the ability to reason that age of reason that they were so proud of in the time of socrates plato and aristotle was lost to civilization for centuries as people just lived by instinct and didn t leave much of a record of their rationale and didn t develop science technology or thought and at a certain time this information that was preserved in the documents of the classics both biblical and religious information and any document that the monks and scribes could get their hands on they preserved and they analyzed it and they studied it and they took a continent and taught that continent how to think as the church emerged from rome and from the st patrick side of this thing out of ireland they built monasteries across the continent and they were the centers of knowledge they began to educate the classical information that they had preserved primarily from the roman but also from the greek era and they reeducated an entire civilization and re created civilization based on what judeo christian values the age of reason and that reason that tied the values of faith together with the values that will allow for science to be developed and that brings us to that year let s say the years emerging from the middle ages and martin luther stepped on to the scene in the 16th century and brought us on top of that the reformation period where he made the point that cast across the globe that you can honor god in a lot of ways a mother changing a baby s diaper honors god more than a thousand rote prayers that you don t give meaning from your heart into and so the protestant work ethic got added to all these values that have been added together and the competition between the protestant and catholic church within christianity ended up it was rough and it was brutal but the effect of it on our civilization and on our society has been good because the competition that drove from that made us all better and each religion drew from the other and by the way the eastern church was separated when the turks sacked constantinople so the eastern orthodox and russian orthodox were separated and they evolved in a little bit different way but we re tied together we re tied together culturally we re tied together historically we re tied together by our common humanity and our belief in and this is the unique component their belief in redemption these attributes that i ve discussed now they re embodied within western europe as we emerge into as we had emerged into the age of discovery meaning christopher columbus and the explorers who came over here to the western hemisphere that component as well as a little bit later the dawn of the industrial revolution think about where we are here in america we are the recipients of some of the wisest most analytical people that the world has ever produced our founding fathers they are a product of a culture and a civilization that believed in adam smith s free enterprise and the rights to property and they believed they were free men that they were free in fact they said so in the declaration independence when jefferson wrote in the declaration a prince who exhibits the characteristics of a tyrant is unfit to be a ruler of a free people a free people they saw themselves as a free people before the declaration they didn t become necessarily free people as a product of although they certainly had to earn it they declared their freedom from england but they saw themselves as free people before they issued the declaration of independence but that brings us now to july 4 1776 i brought this history around from a couple thousand years or a little bit more more than 2 000 years on this continent now we have the wisdom of the founding fathers i believe they are inspired by god and it was by divine guidance that the declaration was written but it arrives here this with what these rights that we have freedom of speech religion the press and assembly the right to keep and bear arms the balance of these rights from the judicial side of it the property rights from the fifth amendment the devolution of power down to the people or the states all of this landed on a continent with unlimited natural resources so we believed at the time all of these rights free enterprise strong judeo christian faith and values the reason many came here unlimited natural resources and a concept of manifest destiny now who could create a giant petri dish that s so robust that it could settle a continent in the blink of a historical eye and leave such a foundation for the growth of population and the image and inspiration of faith and freedom who could do that not man but the entity that shaped their movements and their thoughts so here we are the recipients god given liberty defined in the declaration it should be inarguable it should be unchallengeable i think it is but we re a nation that cannot be reverse engineered and come up with a better result we re a nation that has components of american exceptionalism pillar after pillar of american exceptionalism none of which can we pull out from underneath the edifice of this shining city on the hill and expect that this shining city would not collapse yes it would and so what is our charge here it is not as hard as the charge of our founding fathers it is not as hard as those who picked up their muskets and marched into the red coats muskets and the revolution it is not as hard as the blue and the gray that clashed all over the battlefields here in this country and put an end to slavery and reunified this country it s not as hard as the doughboys that marched off to war it s not as hard certainly as those 16 million americans who put on uniforms to defend our country in the second world war it is certainly not as hard for us as the 450 000 who gave their lives during that war it s not as hard either as those who marched off to korea and are honored down here in their memorial the memorial that says on the slab in front of them our nation honors the men and women who answered the call to defend a country they never knew and a people they never met none of what we are charged with right now is that hard and yet some despair and some think that we can create this new america that is not tied to the pillars of american exceptionalism we can sacrifice some of those principles and we ll still be a country okay because we ve got some political pressure that says we should sacrifice this principle or we should chisel away some pieces out of this beautiful marble pillar of american exceptionalism imagine what it would be like which if this congress and this culture that directs this congress what if we decided you re going to have limited speech certain things you can t say and we ll give you the list of words you can t utter because if you do that you re going to be violating somebody s sense of political correctness what if we said that you can assemble but we re going to diminish your right to assemble because sometimes we disagree with what comes out of those meetings you know the greeks did that they had meetings in their city states remember the greek black ball system that they had the demagogues would emerge people that could step up before the masses in greece and the city state and issue a speech that was rhetorically so inspiring that the greeks marched off in what turned out to be the wrong direction and what would they do they would label him a demagogue they would bring the demagogue before the city state and then they would excoriate him and then they would have a vote it s like the greek system today two gourds two marbles one black and one white one they called them balls of course as each of the greeks walked through they would drop their voting ball in one gourd and they would drop their discard ball in another gourd and if the demagogue got three black balls he was banished from the city state for 7 years that s how they muzzled the people that led them in the wrong direction with emotional rhetoric but can you imagine if we did that if america would banish people into the hinterlands for let s say giving a speech that was disagreed with by three people that s all it took three they were restrained of course because they didn t want to be the next one banished but that was the system we re not going to limit freedom of speech in this country and we re not going to limit freedom of assembly we re not going to say you cannot get together and talk about these things because we know that an open public discourse and dialogue what emerges from that are we believe in this reason that we have inherited from the greeks and other civilizations that what will emerge is the most logical rational policy that s what i m advocating for mr speaker i want the most logical rational policy and i think we need a policy that s right for america i have an obligation to preserve protect and defend the constitution of the united states and represent my constituents and represent my state and represent my country and all of those things should be compatible with each other and i believe they are and i ve not found myself in a conflict here between them so i suggest that we have open dialogue we have open debate i challenge this civilization to be reasonable have reason be analytical be a critical thinker we send our kids off to school and sometimes they re just taught a mantra but they re not taught to take ideas apart and understand the components of them and put them back together well i ve just taken america apart and described some of its essential components history apart and put it back together mr speaker and hopefully informed this body of some of the principal reasons why america is such a great nation we re a great nation because we have god given liberty we would not be a great nation if we didn t exercise those god given liberties if we don t have access to those rights if we don t put our positions out there in front of the public and challenge the people in this country to analyze those alternatives what if we went down one path what if some leader from on high let s just say king george not prince george today but king george what if he decided we re going to go down this path and no one shall discuss anything outside of this line that i ve described for you what kind of a country would we be would we believe that one mortal individual can chart a path for this country superior to the collective wisdom of 316 million people i don t think so mr speaker and i don t think thinking americans will either but i know that this country s full of emotionalism as i watched the reactions to the george zimmerman trial and verdict i saw a lot of people who simply denied the facts that had been proven in law and seemed to be incapable of considering anything that didn t concur with their conclusion that they had drawn before they saw the facts now i engage in this debate i challenge people to debate with me because i believe one of two things if i can t sustain myself in debate i need to go get some more information i need to get better informed or could it be that i m wrong only two alternatives can come from not being able to sustain yourself in a debate and i ll go back and get all the information that i can get but i ll also reconsider and anybody should that s why i challenge people to debate i ll take it up and we will see who can sustain themselves we may not get this all resolved in one discussion in fact in this congress it s been a very rare thing over the last 10 plus years that i ve been here to see anybody stand up and admit i was wrong what you said changes my position what i learned changes my position no there are too many egos involved in this congress for that to happen very often it will happen a little bit privately it will happen incrementally but it doesn t happen publicly unless there s some kind of leverage brought to bear so here s my point mr speaker and that is this our southern border is porous it s not as porous as it was 7 or 8 years ago mainly because the economy has grown in mexico at about twice the rate that it s grown in the united states over the last 4 1 2 or 5 years we don t have as much pressure on our border but i can tell you this 80 to 90 percent of the illegal drugs consumed in america come from or through mexico i can tell you that in mexico they are recruiting kids to be drug smugglers between the ages of 11 and 18 they have arrested and i believe incarcerated and the number of convictions is at least this over 800 per year over the last couple of years at that ratio of those who are kids who are smuggling drugs into the united states we pick up some on our side of the border that adds to that number the ones that we catch many get away every night some come across the border smuggling drugs across the border increasingly the higher value drugs heroin methamphetamine cocaine in some form or another are being strapped to the bodies sometimes of young girls teenage girls the media is replete with this anybody that reads the paper should know especially those that live on the border should know that there are many many young people coming across the border unlawfully who are smuggling drugs into the united states they should also know that now the drug cartels and i mean specifically the mexican drug cartels have taken over drug distribution in most of the major cities in america i think intel will tell you every major city in america and the numbers that i ve seen go from a little over 200 cities in this country to 2 000 i don t know what population that dials it down to or what areas i haven t seen the map but it should be appalling to a country and a civilization to see that that s taken place when you understand that according to the drug enforcement agency of every chain of illegal drug distribution we have in the country they will tell you at least privately as they have to me on multiple occasions that at least one link is illegal aliens that are smuggling drugs into the united states it s important that we know that as a congress as a country as a civilization if we deny those facts if we deny the information that comes even out of the obama administration that certainly supports those if you deny the information that comes out through the major media that s there if you deny what we re told by our law enforcement officers on the border of the united states that are continually interdicting drugs at about the same rate that they did 6 or 7 or 8 years ago when the population of illegals was flowing over the border at a faster rate than it is today the illegal drugs coming across the border are roughly similar to that time that says there s still a high demand in the united states a high demand means drugs are likely to come in if we are enforcing our borders and tightening security the price of drugs should go up if you look at the price of drugs i think you re going to find that we haven t been very effective interdicting drugs coming across our southern border part of that is they find new ways to smuggle and some of those reasons are because kids are being used to smuggle drugs into the united states that s appalling to me the death across the arizona border it s still there it happens through the summer and this debate taking place now in the middle of the summer is going to end up with more people being found out there on the desert in the brush who have lost their lives trying to get into the united states of america we need a secure border we need to build a fence a wall and another fence so we ve got two patrolling zones we need to put the sensory devices on top of there we need to use our boots on the ground in the most effective way possible no nation should have an open borders policy no nation should have a blind eye policy towards the enforcement of the laws no nation can long remain a great nation if they decide to sacrifice the rule of law on the altar of political expediency no nation like the united states of america can continue to grow and be a strong nation if we are going to judge people because they disagree with our agenda rather than the content of their statement we have to be critical thinkers we have to be analytical we should understand facts from emotion and let s pull together let s understand that we do have compassion we do have compassion for every human person deserves dignity we need to treat them with that warmth treat them with that love as the american people always have just like the korean war veterans did when they gave themselves for a country they never knew and a people they never met but we must not sacrifice the rule of law on the altar of political expediency mr speaker i yield back the balance of my time\pagebreak

\section*{support}
mr chairman the purpose of this amendment is simple to prohibit any federal funds from flowing to law enforcement organizations that engage in any form of racial ethnic or religious profiling it s been a matter of concern for decades among minority communities when policing organizations engage in profiling but recent events have brought the problem into sharp focus starting last august the associated press published a series of disturbing stories about the systematic racial ethnic and religious profiling conducted by the new york city police department against muslim and arab americans in new york new jersey connecticut pennsylvania and louisiana in september of last year i asked the department of justice to investigate what we now know was a pattern of surveillance and infiltration by the new york police department against innocent american muslims in the absence of a valid investigative reason these muslim communities were mapped infiltrated and surveilled simply because they were muslim profiling is wrong profiling on the basis of the race ethnicity and religion is a violation of core constitutional principles profiling is also wrong because it is not good policing profiling is an unthinking lazy unprofessional approach to police work and intelligence work and it only raises the risk that the real plot will slip through the cracks indeed profiling is counterproductive the sloppiness of the nypd surveillance effort was such that several non muslim establishments were labeled as being owned by muslims and contrary to the blanket assertions by some that the tactics have kept new york city safe the nypd failed to uncover two actual plots against new york city those perpetrated by faisal shahzad and najibullah zazi in shahzad s case the fbi was surveilling both the mosque he attended and the muslim student association of his accomplice in zazi s case the nypd actually took actions that let zazi be tipped off about the fbi s investigation the nypd s surreptitious uncoordinated and unprofessional approach to counterterrorism prevention within the american muslim community shows that they have learned nothing from the lessons elucidated from the 9 11 commission s report now let me be clear this amendment is not aimed solely at one particular law enforcement organization over the decades law enforcement agencies across the country have profiled against african americans hispanics and other minorities indeed the department of justice has specific guidance prohibiting this practice because it has become widespread and it has conducted litigation against police departments for using race or ethnicity to target citizens for arrest in california pennsylvania illinois and other states my amendment would ensure that no federal funds are flowing to any law enforcement entity that the department has identified as engaging in racial ethnic and religious profiling racial ethnic and religious profiling by police is not something taxpayer dollars should be spent for i urge my colleagues to {\bf \color{red} support} this amendment i yield back the balance of my time people for the american way washington dc may 9 2012 u s house of representatives washington dc dear member of congress on behalf of the hundreds of thousands of members of people for the american way i urge you to {\bf \color{red} support} representative holt s amendment to h r 5326 the commerce justice science and related agencies appropriations act 2013 a vote is anticipated this afternoon this amendment would prohibit federal funds made available through the act to be used for programs or activities that involve racial ethnic or religious profiling by any federal state or local law enforcement organization such profiling undermines america s status as a nation founded on equal justice under law the story of america is one of a nation founded on timeless ideals of liberty and equality and struggling generation after generation to make those principles real for those not included society s outsiders are brought in and made to know that they in fact belong to the community that is america profiling damages that process it sends a powerful message to entire communities that they are in fact not quite the equal members of society that we said they were it tells them that their very existence raises suspicions it harms the individuals profiled as well as those who live in constant apprehension of being profiled the practice undermines our nation s principles and our federal government should not be funding it profiling does not even produce the benefits that it is purported to provide it is counterproductive when limited law enforcement resources are spent targeting innocent people simply because of their real or perceived race ethnicity or religion that is not an efficient use of resources nor is it efficient to alienate entire communities making them feel resentful toward or fearful of law enforcement people living in america should be able to rely on law enforcement as a partner in making their lives safer but those who feel unfairly targeted by profiling will be far less likely to cooperate with law enforcement when their cooperation is needed whether it is a case of local violent crime or national security that does not make our nation or our communities safer a practice that undermines both our principles and our safety is not one that the federal government should be funding we urge you to vote for representative holt s amendment sincerely marge baker executive vice president for policy and program paul r gordon senior legislative counsel
\vspace{8mm}
mr chairman here are a number of organizations that have written in {\bf \color{red} support} of this legislation on both sides of these pages and at the appropriate time i too will insert them in the record to show that there is broad broad {\bf \color{red} support} for this legislation i am now pleased to yield 1 minute to the gentleman from california mr royce
\vspace{8mm}
mr chairman first of all i would like to commend chairman frelinghuysen whom i ve enjoyed working with both here and on the defense subcommittee and ranking member visclosky on their efforts to continue in the tradition of bipartisanship and cooperation i know that all members of the energy and water subcommittee in addition to the staff have worked hard to bring this bill forward and get us where we are today and i want to commend our chairman mr rogers for again presenting us with an open rule which allows the members to have a chance to offer amendments in an era when we don t have earmarks it is very important that members have an opportunity to come here to the floor and offer an amendment i m not trying to encourage anybody but it is a reality now despite the decision made by the republican leadership unfortunately to abandon the overall spending level contained in the budget control act agreement reached last year i m encouraged that this bill provides funding above last year s level the reality however is that if we do not return to the overall levels we agreed to in august proceeding with additional appropriations bills here in the house will be exceedingly difficult many programs in the energy and water bill are sufficiently funded however i do have concerns about the funding levels provided to certain accounts of particular concern to me are deep cuts in the energy efficiency and renewable energy program as well as steep reductions in the arpa e program these programs are vital to continue our nation s innovation in the energy sector i would also like to reiterate mr visclosky s concern over the funding levels of the army corps of engineers relative to fy12 particularly as the corps struggles with its aging structure the bill provides the corps with 188 million less than 2012 we must invest in our infrastructure by making preventative and proactive investments although this subcommittee mark does not fully fund the budget request for the clean up at the hanford nuclear site in washington state i understand that the funding level is sufficient for continued progress and a realistic work schedule for fy13 i want to applaud the chairman and ranking member for continuing the funding for the yucca mountain nuclear waste storage facility during the amendment process of this bill i expect to join an effort led by chairman shimkus to increase funding in this account in order to underscore the strong bipartisan {\bf \color{red} support} in the house for moving ahead with the plan to open the nation s high level waste storage facility i believe as many do in the house that the administration s position to close the yucca mountain site runs counter to the letter and spirit of the nuclear waste policy act passed by the congress\pagebreak

\section*{recorded}
my vote was not {\bf \color{red} recorded} on rollcall no 64 on h r 3036 9 11 memorial act i am not {\bf \color{red} recorded} because i was absent due to the birth of my son in san antonio texas had i been present i would have voted aye
\vspace{8mm}
i demand a {\bf \color{red} recorded} vote
\vspace{8mm}
last month two scientists from oregon state university shawn marcott and alan mix published a peer reviewed study in collaboration with scientists at harvard reviewing 11 300 years of global temperatures they found that the range of temperature change in the last 100 years is equivalent to the temperature change over the previous 100 centuries climate change is real it is devastating and it is accelerating most focus is on the terrestrial effects other research points to rapid and devastating changes in our oceans again a study done by oregon state university burke hales an osu chemical oceanographer coauthor with alan barton who works at the whiskey creek shellfish hatchery looked into the fact that oysters were failing at an incredible rate to spawn and reproduce their study linked the production failures to the co2 levels in the water that has incredible implications for the future of not only the shellfish industry an important industry in the northwest and other parts of the country but also for the whole ocean food chain the ocean chemistry is also threatening something called pteropods who are tiny sea snails and they re very much at risk they happen to be a food source for zooplankton whales and of course our salmon who already have a host of problems in terms of their future then from the arctic monitoring and assessment programme the arctic seas are becoming rapidly more acidified it turns out that cold water is especially susceptible and as the sea ice in the summer recedes more and more of the arctic ocean is exposed to the increased levels of carbon dioxide and it is rapidly acidifying in addition to which the melting of the ice in greenland and elsewhere is adding fresh water which further degrades the capabilities of the oceans to deal with the carbon dioxide finally research in the northeast shows that the surface temperatures in the northeast continental shelf in 2012 were the highest {\bf \color{red} recorded} in 150 years of record keeping they found that over the last four decades many species of fish stocks have been moving north to escape the warming waters but there are many species that cannot move or evolve that rapidly which portends for more disasters back in 1973 there was a science fiction movie called soylent green sort of a mystery movie but it was about an overpopulated and polluted world and the final devastating blow was that the oceans were dying now we have evidence that our oceans are very very much at risk from co2 and climate change the house republicans are using their leadership here to stymie efforts to even research and document climate change let alone just totally denying that it s a problem time and time again they voted to know nothing and do nothing about climate change they voted to block action on climate change no fewer than 50 times in the last congress mr speaker it s time to listen to the scientists and get serious about climate change the evidence is in the only question now is whether congress will listen and act\pagebreak

\section*{extraneous}
i ask unanimous consent that all members may have 5 legislative days within which to revise and extend their remarks and include {\bf \color{red} extraneous} material on the bill under consideration
\vspace{8mm}
i ask unanimous consent that all members may have 5 legislative days within which to revise and extend their remarks and include {\bf \color{red} extraneous} material on h r 1104 currently under consideration
\vspace{8mm}
i ask unanimous consent that all members may have 5 legislative days in which to revise and extend their remarks and include {\bf \color{red} extraneous} material on the bill under consideration\pagebreak

\section*{member}
i intend to reserve most of the time for myself but i have shared with the ranking {\bf \color{red} member} of the armed services committee who s done a very good job and had some commitments and i m yielding to some people as a proxy for him but i will begin by yielding 1 minute to the gentleman from iowa mr loebsack
\vspace{8mm}
i thank mr sherman for his comments and again the progressive caucus special order hour tonight is on the subject of the police power in america and its uses its abuses what has been taking place in different parts of the country and we are going to kick off with keith ellison who has been the chair of the progressive caucus and in addition to being a distinguished {\bf \color{red} member} of the congress from minnesota he is the vice chairman of the democratic national committee mr speaker i yield to the gentleman from minnesota mr ellison
\vspace{8mm}
mr chairman i yield myself such time as i may consume i rise today in support of h r 2289 the commodity end user relief act i want to start by thanking chairman austin scott and ranking {\bf \color{red} member} david scott of the commodity exchanges energy and credit subcommittee they have done a tremendous job over the past few months working on these issues they have held three hearings on reauthorization listening to testimony from end users financial intermediaries and even the commissioners themselves without their work we would not have been able to move this bill today h r 2289 the commodity end user relief act does exactly what the name suggests it provides relief from unnecessary red tape for the businesses that make things in our country end users are the businesses that provide americans with food clothing transportation electricity heat and much much more companies that produce consume and transport the commodities that make modern life possible use futures and swaps markets to reduce the uncertainties that their businesses face farmers hedge their crops in the spring so that they know what price they will get paid in the fall utilities hedge the price of energy so they can charge customers at a steady rate manufacturers hedge the cost of steel energy and other inputs to lock in prices as they work to fill their orders the fact is no end user played any part in the financial crisis of 2008 and no end user poses a systemic risk to u s derivatives markets yet as the agriculture committee heard in countless hours of testimony it is now more difficult and more expensive for farmers ranchers processors manufacturers merchandisers and other end users to manage their risks than it was 5 years ago to address their concerns h r 2289 makes targeted reforms to the commodity exchange act that fall into three broad categories consumer protections commission reforms and end user relief title i of the bill protects customers and the margin funds they deposit at their fcms by codifying critical changes made in the wake of the collapse and bankruptcy of both mf global and peregrine financial title ii makes meaningful reforms to the operations of the commission to improve the agency s deliberative process in doing so it also requires the commission to conduct more robust cost benefit analysis to help get future rulemakings right the first time and to avoid the endless cycle of re proposing and delaying unworkable rules finally title iii fixes numerous problems faced directly by end users who rely on derivatives markets from unnecessary recordkeeping burdens to improperly categorizing physical transactions as swaps to narrowing the bona fide hedge definition cftc rules have discouraged exactly the kind of prudent risk management activities congress intended to protect with the end users exemptions in the dodd frank bill these regulatory burdens present challenges to american businesses and will cost them significant capital to comply with unless congress acts to provide the relief title vii of dodd frank sought to require that most swaps one be executed on an electronic exchange to ensure price transparency two be subject to initial and variation margin and central clearing through the lifetime of the transaction to ensure performance on the obligation for counterparties and last to be reported to a central repository to ensure that regulators have an accurate picture of the entire marketplace at any one point in time h r 2289 does not roll back a single core tenet of title vii it does not change the execution clearing margining and reporting framework set up by the act in fact not a single witness who appeared before the house committee on agriculture ever asked us to upend these principles but what they did ask for were fixes to portions of the statute that didn t work as intended to provide more flexibility in complying with the rules when they impaired end users ability to hedge and to bring more certainty to the commission and how it operates that is exactly what h r 2289 provides similar to the cftc reauthorization bill passed by the house with overwhelming bipartisan support last congress the commodity end user relief act makes narrowly targeted changes to the commodity exchange act this legislation offers meaningful improvements for market participants without undermining the basic tenets of title vii i am proud that the committee has again put together a bill that has earned the bipartisan support of our members because it provides the right relief to the right people mr chairman i urge support of the commodity end user relief act i reserve the balance of my time june 8 2015 dear {\bf \color{red} member} of the house of representatives the undersigned organizations represent a very broad cross section of u s production agriculture and agribusiness we urge you to cast an affirmative vote on h r 2289 the commodity end user relief act when it moves to the floor for consideration this legislation contains a number of important provisions for agricultural and agribusiness hedgers who use futures and swaps to manage their business and production risks some but certainly not all of the bill s important provisions include sections 101 103 codify important customer protections to help prevent another mf global situation section 104 provides a permanent solution to the residual interest problem that would have put more customer funds at risk and potentially driven farmers ranchers and small hedgers out of futures markets by forcing pre margining of their hedge accounts section 308 relief from burdensome and technologically infeasible recordkeeping requirements in commodity markets section 310 requires the cftc to conduct a study and issue a rule before reducing the de minimis threshold for swap dealer registration in order to make sure that doing so would not harm market liquidity and end user access to markets section 313 confirms the intent of dodd frank that anticipatory hedging is considered bona fide hedging activity thank you in advance for your support of this bill that is so important to u s farmers ranchers hedgers and futures customers sincerely agribusiness association of iowa agribusiness council of indiana indiana grain and feed association american cotton shippers association american farm bureau federation american feed industry association american soybean association commodity markets council grain and feed association of illinois kansas grain and feed association michigan agri business association michigan bean shippers association minnesota grain and feed association missouri agribusiness association national cattlemen s beef association national corn growers association national cotton council national council of farmer cooperatives national grain and feed association national pork producers council nebraska grain and feed association north american export grain association north dakota grain dealers association northeast agribusiness and feed alliance ohio agribusiness association oklahoma grain and feed association pacific northwest grain and feed association rocky mountain agribusiness association southeast minnesota grain and feed dealers association south dakota grain and feed association tennessee feed and grain association texas grain and feed association usa rice federation wisconsin agri business association\pagebreak

\section*{house}
this is the last hour that congress will meet before the 7 week recess that the republicans scheduled for today we are going to devote this last hour to focus on an issue incredibly important to the communities of the people we represent and to this country and that is the issue of gun violence as you may recall mr speaker we had a sit in where we came to the {\bf \color{red} house} floor to protest the congressional inaction in moving forward on sensible gun safety legislation to bring attention to break through this logjam and force our colleagues on the other side of the aisle to bring these bills to the floor for an up or down vote we tried motions to recommit and efforts to add these pieces of legislation to bills that were moving as amendments and every mechanism we could to try to force some action because the american people are demanding action asking demanding that we do something in the face of the epidemic of gun violence in this country we talk a lot about gun violence but i think it is important to recognize this is a uniquely american problem we kill each other in this country with guns 297 times more than japan 49 times more than france and 33 times more than israel just to give you some comparisons every day 297 people in america are shot with a gun and each day 89 of these people die on average 31 americans are murdered with guns every day and 151 are treated for gun assault in an emergency room thirty thousand americans die every year at the hands of a gun and the united states firearm homicide rate is 20 times higher than the combined rates of 22 countries that are our peers in wealth and population so it is important as we make this final plea to understand that this epidemic of gun violence is a uniquely american problem we just marked the day before yesterday the 1 month anniversary of the assault in orlando at the pulse nightclub that took the lives of 49 young people we just marked the horrific occurrence in dallas that took the lives of five american heroes dallas police officers it feels like every day there is another mass shooting or a gun tragedy that we hear about and read about in this country what we ask the republican {\bf \color{red} house} leadership is to bring two bills to the floor there are i think 217 bills in total that will respond to gun violence in a variety of different ways but we said let s start with the easy pieces of legislation legislation that is widely supported by the american people that will make a real difference in reducing gun violence in this country and keeping guns out of the hands of people who shouldn t have them that is universal background checks to make sure that someone doesn t get a gun who is not permitted to have a gun under our laws and keeping them out of the hands of domestic abusers criminals and suspected terrorists the second one is the no fly no buy it says look if you are on a terrorist watch list and we have determined you are too dangerous to get on an airplane then you are certainly too dangerous to go into a gun store and buy any gun you want so those two pieces of legislation which are really common sense would be an important first step to demonstrate to the american people that we understand our responsibility to take some action to reduce gun violence in this country and to keep guns out of the hands of people who should not have them rather than taking up those bills regrettably our colleagues on the other side of the aisle adjourned and they went flying out that door so they could go home and enjoy a holiday in the summer with their family and friends without ever taking up a single piece of legislation to address gun violence we tried in every way to say to our colleagues bring these bills to the floor for a vote if you don t support them make your arguments against them let the american people hear you defend that we shouldn t have universal background checks and that it is okay for someone on the terrorist watch list to buy a gun but come to the floor make your argument and vote that is what we get sent here to do give us a vote instead they went out that door and by doing so by failing to act they dishonored the memory of the thousands and thousands of americans who have lost their lives to gun violence mr and mrs wright maria and fred wright were here the day before yesterday on the 1 month anniversary of orlando they came to the capitol rather than spending time at home continuing to grieve about the murder of their son jerald at the pulse nightclub they came here to talk to members of congress they wrote an op ed that was published on the day of their visit to washington they said while in d c we don t want just thoughts and prayers from members of congress we want them to look us in the eyes and tell us how will they work to make our nation safer against gun violence how will they perform their constitutional duty to insure domestic tranquility and promote the general welfare some of the main roles of government according to our constitution how will they work to stand up to the extremist gun lobby and urge their fellow members to do the same that is what they wrote look in our eyes they lost their son and what congress did regrettably is nothing they recessed for 7 weeks mr speaker we have a moral obligation to protect the lives and well being of our constituents that is our most sacred responsibility as members of congress we do that in a variety of different ways we do that by responding to public health crises like the zika virus which we also failed to do we do that by making sure people can have safe drinking water in places like flint and cities all across this country which we failed to do we do that by protecting our constituents from the ravages of gun violence in this country and we did nothing we have a responsibility as members of congress when faced with these sorts of epidemics to do something people who are living in communities all across this country who are living with the consequences of this gun violence say what are you doing to stop it they know we can t pass one law that is going to stop everything but taken together we can pass legislation particularly these two bills that will substantially reduce the likelihood that dangerous people will get guns and harm the communities we represent i will continue to add my voice to this fight as i know many members of our caucus will mr speaker i yield to the distinguished gentleman from california mr thompson he really has led our effort as the chair of the democratic caucus on gun violence prevention and someone who has been a great champion in this effort
\vspace{8mm}
{\bf \color{red} house} republicans have demonstrated their complete disregard and contempt for women s health and the plight of students by forcing a choice between the elimination of funding for the prevention and public health find or relief for students who are saddled with student loan debt that is a choice that we shouldn t and don t have make it is cruel and destructive it is anti family it is not smart economically and it is completely unnecessary as a mother and a grandmother i simply cannot understand why congressional republicans continue their assault on women s health i cannot understand why they prefer to reduce access to cancer screenings and immunizations rather than asking big oil to give up their subsidies i cannot understand why they are trying to force us to choose between keeping moms healthy or sending their children to college if we want to revitalize our economy and unburden americans who are saddled with student loan debt we must enact policies that help to cut that debt democrats have been demanding action on student loans for months and finally republicans have agreed to do something but at what cost by putting the health of women and children at risk the prevention and public health fund supports proven prevention activities like breast and cervical cancer screenings it helps provide immunizations for children it will save lives and keep women well republicans are telling us that we have to choose between protecting women s and children s health or letting student loan rates double republicans are trying to label the prevention fund as a slush fund americans know that mammograms and pap smears are not slush they are basic routine and often life saving services for women prevention is fundamental it is the key to reducing health care costs and creating a long term path to a healthier and economically sound america cutting prevention programs like breast and cervical cancer screening now will only lead to increased health costs down the road in fact the data proves that we should be increasing our investment in early detection through screening and working to increase awareness about these diseases the national health interview survey from 2010 found that women are getting screened for breast and cervical cancers at rates below national standards the breast cancer screening rate was 72 percent in 2010 below the federal health target of 81 percent the cervical cancer screening rates were 83 percent below the 93 percent goal the screening rates for both cancers were significantly lower among asian and hispanic and women as well as those without health insurance or no usual source of health care in the united states in 2012 it is estimated that there will be 226 870 new cases of invasive breast cancer and nearly 40 000 women will die from the disease an estimated 12 000 women will be diagnosed with cervical cancer and over 4 000 women will die from cervical cancer earlier this week republicans on the energy and commerce committee approved over 97 billion in cuts to public health programs to insulate the department of defense from spending cuts triggered by the failure of the joint select committee on deficit reduction among the suggested cuts was the complete elimination of funding for the prevention and public health fund i offered an amendment to preserve support under the fund for breast and cervical cancer screening programs and other women s health preventive services my amendment was defeated along party lines republicans could ask millionaires and billionaires oil and gas companies making record profits and corporations that shift jobs and profits overseas to help offset the cost of reducing student loan interest rates instead they have decided to continue with their repeated war on women s health by eliminating funding for the public health programs that benefit women to reduce the costs for their sons and daughters to attend college
\vspace{8mm}
i yield myself such time as i may consume mr chairman i stand in opposition to this budget i am proud of the fact that we are actually debating a budget for you see you look over to the other body you look to the united states senate and you ll see it s been more than 1 050 days an exceptional amount of time years in fact since the united states senate has actually discussed a budget and here we are debating a budget there s a contrast in vision there s a contrast in priorities but we re debating this on some issues there is some common ground but on other things there is a divergence in our approach this budget that s being presented here as an amendment raises taxes by more than 6 trillion mr chairman let me put in context what 1 trillion is if you spent 1 million a day every day it would take you almost 3 000 years to get to 1 trillion so what we have to have is a realization of the fiscal woes that we face ourselves i didn t create this mess but i am here to help clean it up the reality is we cannot face tens of trillions of dollars in debt because there s a consequence of that the consequence of this massive debt rising interest rates devaluation of the dollar there s so many things inflation as you throw more money into the marketplace imagine what this world would be like if we didn t have what will be at the end of this year nearly 16 trillion in debt right now we re paying more than 600 million a day in interest on that debt so while i think there is common ground and appreciation of what needs to happen for our kids and our future and investments that we do need to make what they would like to do in terms of infrastructure and roads and all of these types of things and our military we re saddled with a 16 trillion debt so we don t have that 600 million we really don t get anything for that we have to pay that as interest on the debt that s where you see a contrast what is being proposed here versus what the republicans are offering in their budget which has passed through the budget committee is they would have to spend 5 3 trillion more over 10 years than what we have proposed so i stand in opposition to this i appreciate the passion and commitment they have to their agenda but i do want to recognize and i hope we can applaud on both sides of the aisle at least here in the {\bf \color{red} house} of representatives we re actually debating a budget with that i reserve the balance of my time\pagebreak

\section*{bipartisan}
i thank my chairman for yielding me the time i came down here to talk about tax policy and my support for the rule mr speaker but i ve got to tell you when folks back home ask me what s wrong with this place i m going to start playing them a clip of this debate because there s a serious topic on the floor right now this fiscal cliff i don t think there s a man or woman in this room with a voting card who doesn t believe this is a serious issue for our economy for working families and for small businesses that we re counting on bringing us out of this recession i believe every man and woman in this room believes that and yet as we re down here trying to have that discussion in the short 11 days we have left to sort that out i hear that our tax package which does exactly what the president has asked though not the levels that he asked for it it picks winners and losers he campaigned on that platform i think it s wrong i think we ought to keep tax rates low for everyone but the president says no the president says we ought to pick some folks who win and some folks who lose and this tax bill does that but it just deals with taxes because as my friend from massachusetts reminded me when i ran as a part of this freshman class i said let s try to make things more simple here because we all know what happens at the end of the year anybody who s watched this process in december knows those christmas tree bills that come rolling to the floor where you handle 100 different unrelated things at one time well mr speaker i d be interested in polling folks who don t have a voting card i d be interested in knowing what folks who ve listened to this debate believe is happening in this underlying tax bill because i ve been told by some of the speakers on this floor that this tax bill throws americans off unemployment when in fact it does no such thing no such thing do we need to deal with unemployment yes we do in an unemployment bill i ve been told that this tax bill cuts payments to doctors it does no such thing there s not one line in this bill that does any such thing do we need to deal with medicare and sgr of course we do do we need to jumble all of these things together in a straightforward tax bill the answer s no i m told by my friend it s not just stony silence on these issues it s stone hearted to be silent who is it mr speaker who believes it advances the debate this hard complicated debate we have who believes we advance it by calling the absence of a nongermane provision stone hearted on the part of the authors don t tell me about violating trust don t tell me about how it is folks ought to work cooperatively together we have that opportunity right now and folks are throwing it away line by line by line my friend from the rules committee comes to the floor mr speaker and he says this bill throws folks off food stamps nonsense nonsense every single time i go to the town hall meeting mr speaker folks believe if only we eliminate the fraud in government we ll balance the budget now due to spending that both sides of the aisle are responsible for we re way far out of balance fraud won t do it mr speaker that s not going to be enough but what the underlying bill does to request to eliminate the defense sequester cuts that president obama s secretary of defense has called so dangerous it says the only people who should get food stamps are people who qualify for food stamps that s right the underlying bill says the only folks who should get food stamps are those who qualify for food stamps now it turns out mr speaker like every federal program there s some fraud and so some folks are receiving taxpayer sponsored benefits today who have not earned them who do not find themselves entitled to them by virtue of their circumstances and because this underlying bill aims to eliminate that fraud folks come to the floor and say why in the world are republicans throwing hungry people out during christmas it s outrageous mr speaker that we can t have a conversation about serious things in a serious time the outrages that my colleagues on the rules committee point to from last night i tell you mr speaker what happened last night is exactly what i would hope would happen in a conversation like this almost to a person every democratic member in that rules committee and those testifying said all we have in front of us tonight is a tax bill all we have in front of us is a tax bill and every american knows the problem isn t taxes the problem is too much spending where are the spending cuts and so the rules committee staff went to work immediately mr speaker and found a package not that had never been seen before not that had never been read before not that had never been vetted before but one that had passed this body in a {\bf \color{red} bipartisan} way they said you know what the criticism from my colleagues is right we do need to do this and we did
\vspace{8mm}
thank you for the recognition i thank the gentleman from washington the chairman of the committee for his recognition and his leadership on this effort as well as dr john fleming the subcommittee chairman and all the members of the house natural resources committee for their support of this particular piece of legislation i also have to thank members of the sportsmen s community members of the congressional sportsmen s caucus and in particular congressmen duncan from south carolina and wittman boren michaud and bonner and my counterparts in the congressional sportsmen s caucus leadership that would be congressmen ross latta and shuler for all their efforts to help advance this legislation in a {\bf \color{red} bipartisan} effort today i join my colleagues in support of h r 2706 which is the billfish conservation act of 2012 as the chairman has already said the united states is the largest importer of billfish products in the world our populations continue to be affected by foreign commercial overfishing and the importing of billfish only exacerbates the problem that exists today without passage of this bill and strengthening of the current ban of the atlantic caught billfish to include the sale and harvest of all billfish excluding as has been already said on the floor today those fisheries in the state of hawaii and pacific insular area the current ban will continue to be undermined through loopholes that have hurt our anglers and the economy by eliminating the sale in the continental u s passage of this bill will support the billfish population growth a healthy ocean ecosystem and improve recreational fishing opportunities as a result of the increased recreational fishing opportunities this bill provides a huge economic boost to generate billions of dollars through direct expenditures and marine related jobs and sales without placing a burden on the u s seafood market and its consumers i want to urge all my colleagues to support this very important piece of legislation to help conserve a very depleted fish population preserving our nation s fishing heritage and provide for economic growth during a time when our country needs it most
\vspace{8mm}
i yield myself such time as i may consume mr speaker i won t go on too much longer we have had great testimony and offerings today by folks who have been working in a very {\bf \color{red} bipartisan} way on a very key component that has been around for 5 years but will absolutely make a difference in solving missing child abduction cases it is common sense it is {\bf \color{red} bipartisan} and most importantly it will help reunite families with missing children mr speaker i yield back the balance of my time\pagebreak

\section*{affordable}
i rise to give voice to my constituents while i would expect that obamacare s thousands of pages would help at least a handful of people a sampling of mail coming into any office lets me know that help by the {\bf \color{red} affordable} care act is rare steve from greenfield says he and his wife are in good health with current insurance costing 485 a month under obamacare that goes to roughly 1 150 a month a 237 percent increase june from batavia received a letter from unitedhealthcare they are discontinuing coverage for most of her family s doctors and while she says she can handle it it will be a problem for her husband he has stage 4 kidney disease and is on dialysis and will soon not have his doctors don from loveland says if the {\bf \color{red} affordable} care act is allowed to stand my family will have to come up with an extra 6 600 next year we can t afford that mr speaker from what i am seeing stress and anxiety are becoming an increasingly common diagnosis all due to obamacare the web site isn t the only problem mr speaker the law is the problem
\vspace{8mm}
i yield myself the balance of my time i can t help but think back to that september evening in 2009 when the president stood before a joint session of congress after the august recess and made a statement to the nation that yes he was trying to change health care but not to worry that no one who was in the country without the benefit of a social security number would be included in that cost because many people are concerned that the cost for the {\bf \color{red} affordable} care act already high would expand unreasonably if that were to change and the president made a promise to the american people that night mr speaker we have heard a lot of stuff today i really wish there had been that much interest in improving the {\bf \color{red} affordable} care act before it passed the first time we all know the reasons why those improvements were not offered and why we just simply had to have a take it or leave it proposition that was ultimately signed into law mr speaker today s rule provides for the consideration of a critical bill to protect the millions of americans who are facing the loss of health insurance that they were promised that they could keep i certainly thank my friend from michigan mr upton the chairman of the energy and commerce committee for producing this thoughtful piece of legislation mr speaker i ask for its approval by the body
\vspace{8mm}
h res 497 provides for consideration of two bills one of which addresses the country s worsening health insurance situation due to the {\bf \color{red} affordable} care act the other addresses the environmental protection agency s attempts to cripple our economy with costly regulations which have dubious health benefits the rule before us today provides for 1 hour of debate for each bill controlled by the primary committee of jurisdiction the committee made in order every amendment submitted for consideration to h r 3826 the electricity security and affordability act including three amendments offered by democrats and five amendments offered by republicans finally the minority is afforded the customary motion to recommit on each bill allowing for yet another opportunity to amend the legislation this is a straightforward rule for consideration of two very important bills h r 3826 the electricity security and affordability act is a bipartisan response to the environmental protection agency s wrongheaded approach to our energy future it was carefully crafted by democratic senator joe manchin from west virginia and the republican chairman of the energy and power subcommittee ed whitfield from kentucky the bill requires the environmental protection agency to acknowledge within its greenhouse gas regulations that different sources of fuel such as natural gas such as coal require different approaches to the regulatory sphere further it prevents the environmental protection agency from unilaterally imposing new regulations on existing power plants those power plants that are already up and running providing heat to our nation which is currently under the throes of a significant cold snap this limitation exists until congress has weighed in and passed a law specifying an effective date for the regulations to begin finally as is just good government the bill requires strengthened reporting requirements from the environmental protection agency one of the most frustrating parts of the epa s new venture in regulating our existing energy infrastructure is that the epa has actively blocked proper congressional oversight from receiving the science and calculations used in crafting these new costly regulations that simply must end if the environmental protection agency is proposing new regulations because they believe they will truly make americans healthier let them share the data let them share the data with the united states congress so it can be peer reviewed both the energy and commerce committee and the science committee have continually been ignored when requesting such data that is unacceptable that must end this legislation is a step toward bringing accountability to an agency that for too long has run roughshod over our economy the second bill contained in this rule h r 4118 suspending the individual mandate penalty law equals fairness act addresses the disparity that president obama and secretary sebelius have created between big businesses which have been given a reprieve from having to comply with the mandates in the {\bf \color{red} affordable} care act and individual americans who have been given no such help by this president just this week the press reported that the administration will delay yet another provision of the {\bf \color{red} affordable} care act by allowing insurers to continue offering health plans that do not meet the {\bf \color{red} affordable} care act s minimum coverage requirements it is becoming so commonplace for this administration to waive or ignore provisions by their own admission this is their signature law and they continue to waive provisions the american people cannot seem to get an even break and no one even seems to notice anymore there is little doubt that this is exactly what the president is hoping for in the last 8 months the president has delayed or modified overly 22 provisions in his signature health care law we are all familiar we have all seen the headlines delays in the preexisting program delays in the employer mandate delays in the reporting requirement changing the rules under which congress has to buy insurance delay delay delay in his own law the president has been quick to fix parts of the law that have political consequences for his allies and to protect his own talking points yet where is the president s protection for the american people under the health care law americans who don t have health insurance and refuse to purchase a government approved insurance policy will face an annual fine an annual fine that increases every year however purchasing a government approved plan also means you have to pay big premiums you are forced to navigate a dysfunctional web site you may lose the doctor you like and place your personal information in jeopardy on an unsecure web site today republicans are offering a legislative solution to help americans get out from under the crushing weight of the so called {\bf \color{red} affordable} care act h r 4118 also known as the simple fairness act will give hardworking americans the same relief that the president has already given to big businesses across the country the administration has no problem delaying the employer mandate not just once for 2014 but a second time for another full year for employers with 51 100 employees shouldn t that same relief be provided to rank and file americans the president has refused to work with congress to change the law so today we are moving ahead and doing what is right for the american people the simple fairness act will eliminate the penalty for 2014 for those individuals who chose not to purchase a government approved health care plan it is clear that h r 4118 offers the only feasible lifeline to millions of americans who are faced with purchasing an expensive health care plan that does not meet their needs it is congress job to protect the american people i urge my colleagues to pass this rule so washington can stop making decisions about american s health care and instead individuals can be free to decide for themselves i encourage my colleagues to vote yes on the rule and yes on the underlying bills i reserve the balance of my time\pagebreak

\section*{ranking}
mr chairman i rise today in support of an amendment i offered along with my fellow colleagues from new york representative sean patrick maloney john katko and jerry nadler this amendment will improve the safety of bridges across new york state and indeed across the nation as you are aware mr chairman our national bridges are in desperate need of repair in new york this is especially true in 2015 the american society of civil engineers graded new york s network of bridges as a dismal d plus new york ranks second worst in the nation in functionally obsolete bridges and 12th worst when it comes to structurally deficient bridges this is not an issue limited to new york across the nation more than one in nine bridges are graded as structurally deficient and more than 84 000 functionally obsolete bridges are still in use mr chairman our amendment does something positive and constructive about it by directing the dot to develop a strategy to address structurally deficient and functionally obsolete bridges notably these two categories require different policy solutions but too often they are treated the same by requiring this strategy we will allow for effective oversight by the people through their representatives here in the u s house i want to thank chairman shuster and {\bf \color{red} ranking} member defazio for their strong work in the committee i urge support of this amendment so we can develop a strategy to address the quality of bridges across this nation which will help keep our people safe and help strengthen our economy mr chairman i yield back the balance of my time
\vspace{8mm}
mr chairman i will be brief and also try to improve just a little bit on a very good bill my congratulations to the team and to the {\bf \color{red} ranking} member and to the chairman included in the u s fish and wildlife resource management account is funding for the national wildlife refuge system by my amendment we ask that an extra 1 million be added to this account we are offsetting the increase by taking 1 million from the 2 4 billion environmental protection agency programs and management account hardly a stretch the national wildlife refuge system has grown to over 563 national wildlife refuge and 38 wetland management districts 150 million acres in all we have several of these national wildlife refuges in my district or near my district including j n ding darling national wildlife refuge on sanibel island and the florida panther national wildlife refuge outside of naples the ding darling national wildlife refuge in particular sets itself apart as a leading contributor to the economy with 816 000 visitors a year importantly my hero my mother loves to go there and i love to take her there in the autumn of her lifetime i ask my fellow members to support this 1 million adjustment to these national treasures
\vspace{8mm}
i thank the chairman for the time i applaud the work that both chairman royce and {\bf \color{red} ranking} member engel have done in helping to focus more of our foreign policy priority here in our own western hemisphere i also applaud congresswoman norma torres for authoring the measure that we have before us today h res 145 reaffirming our dedication to the fight against corruption in central america it is an important measure mr speaker and it is an important fight for years i have been a strong advocate for this fight because where corruption is allowed to spread drug trafficking and crime inevitably thrive and this is negative for our neighbors it is bad for us and it is bad for our interests that is why it is vital that we make battling corruption in the region more of a priority of our foreign policy in fact earlier this year i traveled to honduras and guatemala with my good friend albio sires and we saw firsthand how these governments are attempting to tackle corruption in their countries it is not easy mr speaker they are making progress and taking some of the tough decisions necessary but there is so much more to be done and so much more that they need to do but they need help from the united states that is what we heard when we hosted the attorneys general from the northern triangle countries here in washington d c just last month to discuss what they are doing to fight corruption and what assistance they might need from us that is why this resolution before us is so important and so timely we must urge the governments of central america to do more to battle corruption but we also must pledge to do more ourselves because they cannot do it alone central american governments must take a stand and voice their support for anticorruption programs they must respect and defend the authority of the judicial branch and they must make it a priority that is not easy for them to do some of these governments have shown a willingness to take these steps but sadly mr speaker not all of them have while we urge willing partners to take the steps necessary to fight corruption we must be willing to do more for those unwilling that is why i have reintroduced my nica act which aims at tightening the economic screws on the ortega regime until we see some drastic reforms including efforts to end corruption it is our duty to support our neighbors so that our partners to the south can live in far more open free and democratic societies it is also in the benefit of our security and it is in the benefit of our national interests to do so that is why i urge my colleagues to support h res 145 i also urge my colleagues to support my nica act and to take a more engaged role in our foreign policy interests in our own western hemisphere\pagebreak

\section*{may}
mr chairman before i yield to my colleague from new hampshire i just want to point out that we re not quite understanding the bill here on the other side because we do allow the fcc to maintain flexibility where necessary the bill only requires the notice of inquiry on new rulemakings the requirement does not apply to deregulatory rulemakings and the fcc {\bf \color{red} may} waive the notice of inquiry in emergencies or where conducting both a notice of inquiry and a notice of proposed rulemaking would be unfeasible so we tried to put some balance in here but what s wrong with having the fcc even in that case as raised by mr doyle take 60 days they can decide how long this is and go out survey the market and say what effect and what are the issues and then come back and then they write their rules it s like us having a hearing this isn t a burdensome requirement i yield 1 minute to the gentleman from new hampshire mr bass
\vspace{8mm}
earlier this month my congressional office in titusville pennsylvania participated in a bridge naming service for lieutenant colonel michael mclaughlin of tionesta forest county located in pennsylvania s fifth congressional district thanks to the efforts of state representative kathy rapp the bridge was renamed the lt col michael mclaughlin amvets post 113 memorial bridge lieutenant colonel michael mclaughlin was actually born in germany but raised in forest county he graduated from the west forest high school in tionesta and later attended clarion university it was there he became an rotc cadet and was commissioned a second lieutenant in 1982 starting his military career in the army reserves lieutenant colonel mclaughlin went on to earn a master s degree from the university of pittsburgh and later became the president of his own company in mercer pennsylvania all while serving in the pennsylvania army national guard throughout his service he was highly honored earning many ribbons and medals throughout his 26 years of service unfortunately lieutenant colonel michael mclaughlin was killed in the line of duty on january 5 2006 in ramadi iraq as the result of a suicide bomber he was just 44 years old and left behind his wife and two daughters mclaughlin was honored posthumously with the purple heart and the combat action badge he was the first field grade officer of the pennsylvania army national guard to die in action since world war ii i was proud to see members of lieutenant colonel michael mclaughlin s community come together to honor him with this bridge naming it is so fitting that it came in {\bf \color{red} may} the same month as memorial day when we honor the men and women who lost their lives in service to our great nation i am the proud father of an army soldier america s servicemen and women are very important to me with memorial day coming up on monday i want to not only recognize the sacrifice of men and women such as lieutenant colonel mclaughlin who have given the ultimate sacrifice but all of the members of our armed forces serving across the globe and all of our nation s veterans
\vspace{8mm}
mr chairman i yield myself such time as i {\bf \color{red} may} consume mr chairman i rise in strong opposition to h r 5 a bill to reauthorize the elementary and secondary education act esea a landmark civil rights law enacted under president lyndon b johnson as we approach the 50 year anniversary of its enactment we cannot take lightly esea s mission goals and achievements over the course of five decades it is by that yardstick of history that we must judge h r 5 today and determine if it will move our education system closer to meeting the challenges of the 21st century and prepare our students for the global economy we all know too well that quality education is even more vital today than it was generations ago in our rapidly changing economy our nation s continued success depends on a well educated workforce a competitive and educated workforce strengthens the very social fabric of america people with higher levels of education are less likely to be unemployed less likely to need public assistance less likely to become a teen parent and less likely to get caught up in the criminal justice system over the course of esea s history we have recognized that for many politically disconnected populations equitable access to an education has not been a reality it was necessary for the federal government to fill in the gaps of funding our public school systems inequality was inevitable when most school systems are funded by real estate taxes and further by virtue of the fact that in our democratic society we respond to political pressure for 50 years congress has recognized that low income students were not getting their fair share of the pie and that supplemental resources were absolutely necessary to ensure that all children had access to quality public education as a result congress has a longstanding policy to target our limited federal funding to schools and students who get left behind in an unequal system mr chairman one of this bill s most troubling provisions which strikes at the heart of esea s long history of targeting resources to our neediest students is the so called portability provision now present law gives greater weight to funding in areas of high concentration of poverty under h r 5 portability a state agency could use all of its title i funds to districts based solely on the percentage of poor children regardless of the concentration of poor people in a district as a result much of the title i support intended towards those areas of concentration of poverty would be reallocated to those wealthier areas in other words the low income areas would get less and the wealthy areas would get more i ask if that is the solution then i wonder what you think the problem was analysis from a number of organizations including the department of education demonstrates title i portability will take money from the poorer schools and school districts and give more to affluent districts this disproportionately affects students of color and this is just simply wrong data shows that h r 5 would provide the largest 33 school districts with the highest concentration of black and hispanic students over 3 billion less in federal funding than the president s budget over the next 6 years furthermore the center for american progress found in its review of portability that districts with high concentrations of poverty could lose an average of 85 per student while the more affluent areas would gain more than 290 per student there is an overwhelming body of research that shows that targeting resources to schools and districts with the highest concentrations of poverty is an effective way to mitigate the effects of poverty current law reflects this evidence and targets funding to schools where there are greater concentrations of poverty and this bill rolls the clock back and reverses that to add insult to injury h r 5 eliminates what is called maintenance of effort a requirement of esea that states maintain their effort and that the federal money will supplement what they are doing as a result of this bill states could use their education funds to fund tax cuts or other noneducation initiatives thus turning esea into a glorified slush fund where politics would drive funding allocations and we know who is going to lose when politics are at play our children there are other flaws with h r 5 this bill sets no standards for college or career readiness and allows students with disabilities to be taught with lesser standards it limits our investment in education over the next 6 years because there are no adjustments for inflation it block grants important programs diluting the purpose and the outcome taken as a whole these policies will have a disproportionate impact on students of color students with disabilities and our english language learners it is no wonder that business groups labor groups civil rights disabilities and education groups have all expressed deep concerns about this legislation mr chairman i stand in strong opposition to h r 5 as it will turn the clock back on american public education in its current form the bill abandons the fundamental principles of equity and accountability in our education system it eviscerates education funding it fails to support our educators and it leaves our children ill prepared for success in the classroom and beyond therefore i urge my colleagues to vote no on this bill and i reserve the balance of my time\pagebreak

\section*{material}
i ask unanimous consent that all members have 5 legislative days to revise and extend their remarks and to include any extraneous {\bf \color{red} material} they may have on h r 4057 as amended
\vspace{8mm}
i ask unanimous consent that all members may have 5 legislative days to revise and extend their remarks and to include extraneous {\bf \color{red} material} on the bill under consideration
\vspace{8mm}
i ask unanimous consent that all members may have 5 legislative days within which to revise and extend their remarks and include extraneous {\bf \color{red} material} on h r 4768\pagebreak

\section*{members}
i have come to this floor once a week during the 113th congress to talk about hunger specifically how we can end hunger now if we simply muster the political will to do so technically identified as food insecure by the department of agriculture there are nearly 50 million hungry people who live in the united states the richest country in the history of the world these people don t earn enough to be able to put food on their table simply they don t know where their next meal will come from now let s be clear this has not been a particularly kind congress to those who struggle with hunger we are seeing nearly 20 billion cut from our nation s preeminent antihunger program known as snap snap is a lifeline for the 46 million americans who rely on it to have something to eat each day yet this congress decided that americans who live at or below the poverty line can simply absorb massive cuts to snap sadly republicans and some democrats joined together to cut a benefit that was already meager and didn t last through the month even before these cuts took effect these cuts are bad and hurtful but just as hurtful is how these americans were described and depicted on the floor of this house during the debate about cuts to snap during the debate on the farm bill some republican {\bf \color{red} members} came to the floor to justify cuts to snap as a way to prevent murderers rapists and pedophiles from getting a government benefit poor people have been routinely characterized as those people as part of a culture of dependency they have been described as lazy mr speaker i am sick and tired of poor people being demonized i am sick and tired of their struggle being belittled we are here to represent all people including those struggling in poverty unfortunately insults continue for the most part we try to keep campaign rhetoric out of the debate on the house floor however today i want to highlight some rhetoric that is even more vile than even some of the language that was used on the house floor during the snap debate a few weeks ago a republican candidate for united states senate in south dakota actually equated snap recipients to wild animals that s right we are now at a point where it is apparently okay for political candidates to denigrate our fellow citizens by comparing them to animals dr annette bosworth shared a viral image on her facebook page that said the following the food stamp program is administered by the u s department of agriculture they proudly report that they distribute free meals and food stamps to over 46 million people on an annual basis meanwhile the national park service run by the u s department of the interior asks us please do not feed the animals their stated reason for this policy being that the animals will grow dependent on the handouts and then they will never learn to take care of themselves the post continues this concludes today s lesson any questions what an incredibly offensive thing for anybody to say mr speaker i was taught to love my neighbor i was taught to care about the people and to strive to make everyone s life better and what is being tolerated as political dialogue violates those teachings and my core beliefs in humanity we can all do better some of us may need a hand up in order to get by but that doesn t mean that they are lesser people for it they deserve our respect and they deserve our help while they are struggling it is hard to be poor and because of many of the actions that have been taken by this congress it is even harder to get out of poverty dr bosworth should apologize to the 46 million of her fellow americans who need snap to put food on their tables she should apologize to the nearly 50 million of her fellow americans who struggle with hunger and don t know where their next meal will come from and republicans should repudiate her disgusting remarks i am an optimist i believe we can end hunger and i believe we can end poverty in america if we just make the commitment to do so but hurtful rhetoric like this simply divides us and does nothing to help us achieve the worthy goal of ending hunger now hunger is a political condition we have the food and we have the ability to make certain that nobody in this country goes hungry but we lack the political will and demonizing the poor as so many in this chamber have done and continue to do so is a sad commentary on this congress our government has a special obligation to the most vulnerable it is time we lived up to that obligation the war against the poor must stop
\vspace{8mm}
i rise to salute a caring mayor the mayor of the city of houston mayor sylvester turner over the last couple of months which is the first of his term many challenges have confronted his administration one of which was the terrible devastating floods of mid april when so many thousands many of them mothers and children were displaced in my congressional district because of those terrible storms 400 million of cars were lost people lost their jobs and businesses were destroyed mayor turner continued to be that caring steady hand working across political lines working with the county judge working with council {\bf \color{red} members} and the federal government one thing that he steadily did was listen to the council and the advice of his staff as i sat in meetings taking ideas establishing a relief fund joining now with the osteens in lakewood and having this wonderful concert to continue to provide relief but yet showing the caring and loving nurturing of a father in the midst of all of this he lost a dear brother a vietnam vet but steady strong and determined he continued to nurture those who could not help themselves what a pleasure to be able to work with a mayor one who is ready to listen and to be able to answer the concerns of a constituency but make hard decisions i salute you mayor turner as someone who cares about our city and works with all of us to make their lives better and our city the best
\vspace{8mm}
i yield myself such time as i may consume mr speaker i thank my colleague mr engel for his comments i will return to this theme about the urgent threat that the united states and our allies face here we have listened to experts who have looked at this problem in less than 4 years pyongyang may have the ability to make a reliable intercontinental ballistic missile topped by a nuclear warhead capable of targeting the continental united states when we watch these tests and we see from a north korean submarine how they are launching missiles and we watch the atomic weapons tests that they are doing you can see how north korea has advanced in their capabilities as they try to shrink these warheads and figure out how to put them onto an icbm the problem is that in the next few years at the current rate of production of their nuclear material they are going to be able to build out 100 atomic weapons for these intercontinental ballistic missiles so the threat from north korea is real and real threats demand real responses we have tried various approaches in the past we tried strategic patience during the last administration i will tell you that i think secretary tillerson has helped devise a strategy of maximum pressure that makes a tremendous amount of sense to me and i will share with you why i think it is very credible we have seen in the past in 2005 back during banco delta asia back when north korea was caught counterfeiting 100 u s bank notes a strategy deployed that froze the capability of that regime to move forward with its nuclear weapons program we know from talking to defectors about the impact that that had internally on north korea because frankly these weapons programs are very expensive to run it requires billions and billions of dollars every year north korea doesn t really manufacture much other than some of the clandestine missile parts and so forth that they transfer overseas and some meth and counterfeit cigarettes all of that can be halted so that hard currency doesn t come into the hands of the regime and therefore the regime will no longer have this capability because it happened in 2005 and because we know of the consequences at the time but also because of what we have seen with other nations we should move with bipartisan legislation here i am going to speak for a moment about what this house of representatives and our counterparts in the senate did in the 1990s when it came to the issue of a regime in south africa that had obtained a nuclear weapon and also was doubling down on their practices of apartheid in terms of the way that that regime treated its own people if you will recall despite the assurances and warnings about sanctions that this was the wrong road this house stood up and over 80 percent of the {\bf \color{red} members} here and over 80 percent of the {\bf \color{red} members} in the senate or 75 huge bipartisan majorities of republicans and democrats came together with a policy that said enough enough of the conduct of that apartheid state enough of them developing a nuclear weapon it was time for the united states to lead on this work with the international community and enforce sanctions in a way that did what that within a short period of time brought the apartheid regime to offer up to the international community that atomic weapon and to say we are done with it and for the south african apartheid regime to say in terms of elections next year we are going to hold free and fair elections in south africa in terms of the release of nelson mandela and in terms of his election to president of south africa now when people argue with us that sanctions may not be a way forward i would remind them that when we unite the international community and when we speak with one voice yes we could see a change of conduct in this regime in north korea so i say this gives a powerful tool to cut off the funding by going after those who do business with the regime in violation of u n security council resolutions mr speaker i thank mr engel for his assistance in this and i thank all of my colleagues who helped on this measure mr speaker i yield back the balance of my time\pagebreak

\section*{care}
the house will again vote to address the imprecisely named patient protection and affordable {\bf \color{red} care} act critics say that we re tilting at windmills well mr speaker let s review within the last month or so we ve heard from the senator who authored the law refer to it as a coming train wreck that s right he called it a train wreck we ve heard the administration official responsible for helping set up the insurance exchanges worry that the public might be in for a third world experience as they try and find health {\bf \color{red} care} oh and let us not forget some of the very same members of congress who voted to foist this massive overreach on americans are now feverishly trying to find ways to exempt themselves and their staffs from its effects let s look at the checklist shall we premiums shooting up check small businesses hiring fewer workers and jobs being lost check employees seeing their hours cut check faulty cost projections check everything that opponents of this law listed as a reason to vote against this example of government overreach is actually occurring and happening tilting at windmills mr speaker hardly working to protect the american people from a horribly disruptive and ineffective law certainly
\vspace{8mm}
before i yield to the next speaker on our side i think it is important to point out that yeah the democrats do control the majority in the senate but a majority of republicans right now are filibustering consideration of extending unemployment insurance led by mitch mcconnell the republican minority leader maybe rather than waiting for them we can show some leadership here and demonstrate to these millions of americans who have fallen on tough times that somebody cares that we are not just going to let them just dangle and be without any kind of compensation during these difficult times that we are going to step up to the plate and let them know that we understand that the economy is still going through hard times and that there is a need to extend this benefit i don t know how we can just turn our backs on these people who are struggling i mean our job here is to help people not to ignore their problems not to turn a cold shoulder when they fall on difficult times we all know we are emerging from one of the worst economic crises in our lifetime these aren t normal times so we ought to be there to provide some help let us show them a little compassion i don t think that that is unreasonable i don t {\bf \color{red} care} what your ideology is we ought to not turn our backs on those who are unemployed in this country with that i yield 1 minute to the gentleman from nevada mr horsford
\vspace{8mm}
well here we are the affordable {\bf \color{red} care} act is going to be 3 years old in just a few days and we re continuing to uncover things within the law that nobody knew about remember all the stuff that was sold to the public because it was going to be free but we all know nothing is free so how do you pay for it well it turns out there s going to be tax on insurance companies and taxes on employers which guess what that s going to be passed on to the employees and the beneficiaries the deadline is quickly approaching and plans are submitting their bids but they re faced with no choice but to raise costs in response to the rate increases the federal government is attempting to limit higher premiums by something they call rate review but anytime you treat only the symptom of a disease and not the underlying cause you re going to end up with something you didn t expect continued regulatory pressure continued pressure on employers and continued pressure on insurance plans is going to result in actually further increasing rates the government is attempting to control the market but we all know this market is one the government cannot control and the end result is that we ll all suffer let s face it instead of if you like what you have you can keep it what they really meant to say was you re going to pay a lot more to get a lot less\pagebreak

\section*{include}
i yield myself such time as i may consume i rise in support of h r 4168 the small business capital formation enhancement act i would like to thank the gentleman from maine mr poliquin and the gentleman from california mr vargas for their bipartisan work on this bill i go off script here just to say thank you very much to mr poliquin who has been a very active member on this committee from the very beginning and has been very active in making sure this legislation has come to the floor today i thank the gentleman as i said before this bill came out of committee due much in part to the gentleman s work with an overwhelming bipartisan vote i believe it was 55 1 so the gentleman just has that one to work on for his next piece of legislation that comes out of committee mr speaker congress created the sec government business forum on small business capital formation to do what to provide a platform to identify unnecessary impediments to small business capital formation and to find ways to eliminate or to reduce them each forum seeks to develop recommendations for government and private action to improve and provide the environment for small business capital formation thereby providing small businesses the opportunity to do what to grow economically and most importantly as we have been talking all day to create more jobs unfortunately the sec s default position over these several years has been to simultaneously and summarily ignore many of the recommendations made by the various forum participants which {\bf \color{red} include} small businesses venture capitalists trade association representatives accountants academics and other small business academics despite the claims of which we hear every year from the commission about the importance of this forum it seems that the only time the sec actually implements one of these capital formation agenda items that comes out of it is when congress tells it to do so this was certainly the case with several provisions of the jobs act many of which as one will recall were original recommendations from that very same forum i will give two examples there was the crowdfunding and the regulation a plus provisions of the jobs act they basically mirrored the forum s recommendations years earlier the small business capital formation enhancement act which is before us today provides an answer it basically provides a simple solution to making the sec more responsive it requires the sec to respond publicly and in writing to each forum recommendation and to simply explain whether it plans to take action on that item or not it really shouldn t take an act of congress for the sec to fulfill its basic capital formation mission quite honestly it shouldn t take an act of congress for the sec to simply respond in writing to any of the forum recommendations unfortunately this is the position we find ourselves in today so we have h r 4168 which is the gentleman from maine s work which will ensure that the sec no longer ignores these recommendations and will be able to help fulfill its statutory mission to facilitate capital formation in this country mr speaker i reserve the balance of my time
\vspace{8mm}
i would like to thank congressman raskin for organizing this time for helping to keep this congress focused on this extraordinary chain of events that is taking place in our country and for drawing attention to what should be a credible investigation about the ties between this president and vladimir putin and the kremlin why is this president so focused on complimenting putin why has he wrapped his arms around him why has he said he is a great president why does he refuse to even talk about the fact that putin has invaded crimea why does he refuse to understand what is being said when putin is charged to be a killer and all of the deaths that are taking place from opponents of his from people who criticize him well i think the more we learn about the connections that this president and his allies have the more these questions are going to become very serious and it is going to lead us to have to make some big decisions about whether or not this president is fit to lead the united states of america i have been deeply concerned about these issues for months president trump throughout his campaign and since his election has chosen to surround himself with people who have close ties with russia when our intelligence agencies announced their conclusion that russia interfered in our elections i called for an investigation focusing on the possibility of collusion between trump s kremlin klan that i have dubbed them and the russian government i introduced h con res 15 urging congress to investigate the possibility of collusion between russia and the trump campaign investigations should focus on the kremlin klan let s talk about some of those allies and folks who are aligned with trump and with russia michael flynn who was fired from the nsc after lying about discussing sanctions with russian ambassador kislyak paul manafort trump s former campaign manager was a paid lobbyist for viktor yanukovych the pro russian politician in ukraine who fled to russia in 2014 ap reports manafort signed a 10 million contract in 2006 with russian billionaire and putin ally oleg deripaska to advance putin s interest in the united states the new york times also reports manafort tried to hide 750 000 in payments from a pro russian party in ukraine carter page a former trump campaign adviser is a consultant to and investor in the kremlin state run gas company gazprom and has a direct financial interest in ending american sanctions against the company he recently revealed that he met with russian ambassador sergey kislyak during the 2016 rnc and then there is roger stone who has worked in ukraine stone announced in a speech last summer that he had spoken to wikileaks founder julian assange stone also disclosed to the press that he had been exchanging messages with guccifer 2 0 the russian hacker that hacked the dnc last summer and then secretary of commerce wilbur ross was a business partner of viktor vekselberg a russian oligarch and putin ally in a major financial project involving the bank of cyprus secretary of state rex tillerson signed a multibillion dollar agreement with russia in 2011 on behalf of exxon for an oil drilling project in the arctic and is focused on lifting the sanctions the new york times reported that prior to his resignation mike flynn was delivered a proposal outlining a way for president trump to lift the russian sanctions and broker a deal between russia and ukraine that also included the public smearing of ukraine s current president poroshenko the deal is being pushed by his opposition in ukraine although mike flynn is gone the proposal remains along with those pushing it then there is michael cohen the president s personal lawyer who was involved in developing the document and who delivered the document then there is felix h sater a business associate and a former criminal who served time who reportedly had ties with the mafia who helped mr trump scout deals in russia and then there is andriy artemenko a ukrainian lawmaker trying to rise in a political opposition movement shaped in part by mr trump s former campaign manager paul manafort and of course there is our attorney general jeff sessions who was forced to recuse himself from investigations related to the 2016 presidential campaign after it was revealed that he met with russian ambassador sergey kislyak on two separate occasions during the campaign cycle information which he failed to disclose during his confirmation hearings kislyak is the same ambassador with whom mike flynn discussed u s sanctions and by the way he lied about it it has now been revealed that russian ambassador sergey kislyak met with the following trump associates carter page jeff sessions mike flynn and jared kushner in december 2016 in trump tower during the transition none of these meetings were made public and were only discovered after the press released reports before the press reporting on the meetings above the trump administration had repeatedly denied its campaign had contact and communication with russian officials the press has noted that the meetings are not unusual but that public concern is heightened because they have all lied about or failed to disclose the meetings deutsche bank was ordered to pay more than 600 million in fines including a 425 million fine to new york s department of financial services and a 204 million fine to the u k s financial conduct authority for failing to have adequate money laundering controls in place to prevent a group of corrupt traders from improperly and secretly transferring more than 10 billion out of russia press reports indicate that the department of justice is investigating this matter deutsche bank is trump s largest lender lending his companies an estimated 360 million as to oil and gas president trump signed last month a bill striking section 1504 of the dodd frank wall street reform act which required big oil companies to disclose the money they pay to foreign governments to drill on their lands striking section 1504 will allow big oil companies like exxonmobil to conduct secretive dealings with corrupt parties such as vladimir putin and russia the white house attempted to enlist the fbi the cia director pompeo and top republicans on the house and senate intel committees to help push back against the new york times reporting on trump s ties to russia there is devin nunes i don t need to talk about him he issued a joint statement with adam schiff a joint statement in january announcing that the scope of their investigation would {\bf \color{red} include} links between russia and individuals associated with political campaigns fbi director james comey announced on march 20 during testimony before the house intelligence committee that the fbi is investigating whether members of president trump s campaign colluded with russia to influence the 2016 election devin nunes announced to the press that members of trump s transition team were under incidental surveillance by u s intelligence agencies after the election and briefed president trump on march 22 however he did not brief adam schiff or other house intelligence committee members and he never revealed his source devin nunes has clearly compromised the investigation and can no longer be trusted to lead it in conclusion congress must create a comprehensive independent bipartisan commission to expose the full truth of trump s ties to russia i believe that once we have fully investigated trump s kremlin klan we will find that there was collusion between president trump and russia to violate the integrity of our elections at that point the republicans in congress will have no choice but to put country ahead of party i say impeach donald trump i thank you so much as we witness what attempts to be a coverup now about all of this
\vspace{8mm}
i ask unanimous consent that all members may have 5 legislative days within which to revise and extend their remarks and {\bf \color{red} include} extraneous materials on h r 1428 currently under consideration\pagebreak

\section*{question}
i want to thank the gentlewoman from houston sheila jackson lee for sharing so many great stories and fond memories of speaker jim wright i would like to add a few words of my own jim wright was very influential to me when i was elected into the state legislature in 2004 was when i really started to get to know him well i had known him previous to that when i was an aide for united states congressman martin frost who was also from fort worth once i got into the state legislature i got to know him even more and i realized very quickly what a great storyteller he was speaker wright had some amazing stories from people that he had met over the years people that influenced him in his life so many people always wonder how he became the great orator that he was there were so many stories that i heard early on about before the house had c span now we can watch coverage of the house of representatives 24 hours a day thanks to technology but speaker wright was such a great orator that before c span came into effect you heard stories about staffers coming to fill the galleries so they could come and hear this man from fort worth texas come in and give speeches because they were so amazing i asked him how did you become the great orator that you were when you were in the u s house of representatives and that you still are today even unfortunately with the oral cancer that he had his speech had been hampered but it was still amazing the wisdom and the knowledge that he shared as you have heard from so many speakers tonight boxing was a very important part of life he loved boxing it was something that he watched over the years when he was growing up in weatherford texas that was one of the ways how young boys and men distinguished themselves was their boxing skills on the street he told me that one day his dad told him that while it was great that he was able to distinguish himself with his fists through boxing that if he really wanted to improve himself and improve his lot in life that he would learn how to be a great orator that he would learn what the anatomy of a great speech was all about so jim wright at a very early age decided that he was going to learn how he could become a better speaker and there are so many stories like that i went to his office right before i was sworn in in 2012 and i asked him to just share some of that wisdom with me as an incoming new member of congress he told me so many stories that day one of them related to boxing many of you know larry hagman from i dream of jeannie and from the tv series dallas some of you may know that larry hagman s mother is mary martin of peter pan fame mary martin was actually from weatherford texas and she knew jim wright and knew speaker wright s family i said larry hagman told a friend of mine that he ran into that you taught him how to box is that true speaker wright began to tell me the story about his mother thought that maybe he needed to get back to his texas roots and have a little bit more texas upbringing in him and so she sent him back to weatherford texas with his dad and speaker wright taught him how to box that was how larry hagman learned how to become a boxer one of the areas that speaker wright and nancy pelosi talked about it a lot how he was a big influence in my life and so many others lives and i would be remiss if i did not mention some of the former members that also he was very influential in their lives congressman martin frost who was the ranking member of the rules committee speaker wright was very very influential in getting him on the rules committee his freshman year in office also secretary pete geren a former member of congress and secretary of the army and air force again speaker wright was very influential early on in his career pete geren was actually speaker wright s successor in congress and that was also very important to him many people know that speaker wright was known as a very strong democrat he was someone that loved the democratic party that was very proud of his democratic roots and had a very strong relationship with organized labor in tarrant county when you talk to people that are longtime employees at lockheed martin at general motors at american airlines the things that he did with transportation all of those things were very very important for who he became in addition to that he also learned a lot from some of the failures and mistakes that he made he told me that his first term in the state legislature that it was not easy that he didn t get along with the speaker of the house in the state legislature when he was elected here he wanted to make sure that he got along with sam rayburn when he was elected to congress he told me marc i have learned my lesson from when i was in the state legislature and i really wanted to be on the foreign affairs committee because that was what was really happening back in the 1950s when i first got elected with the cold war going on i wanted to be on that committee it was something very important to me speaker rayburn put me on the public works committee which is now the transportation and infrastructure committee he said that ended up that was a mistake that i made because that committee ended up really making my career it is hard to think that i would have become majority leader and speaker of the house had i not been on the public works committee which is where speaker rayburn put him again in addition to being that strong strong democrat that he was i can tell you that bipartisanship was something that he was very fond of because he talked a lot after his career in congress about how bipartisanship helped make this country strong and about how it helped make him a better member of congress if you go and look in the archives of the star telegram from just a couple of months ago after he passed you will notice the remarks that were given from a very bipartisan group of people in the dallas fort worth area roger williams also from fort worth he was quoted in the star telegram kay granger former mayor of fort worth was also quoted in the star telegram about how speaker wright did so many great things for fort worth one of the areas that he liked to talk about was the voting rights act and how important voting rights were to him and also eisenhower and the freeways he told us a great story about how he and a few other congressmen went to eisenhower about getting the interstate highway bill passed and how president eisenhower said let s get the votes let s get it done and how they came together in a bipartisan way in order to get that legislation done my favorite story that he told me about is the importance of bipartisanship i asked him mr speaker i am going to be a new member of congress and so many people talk about how congress is broken and they don t work together i said do you have any theories on why that is he said that is a very good {\bf \color{red} question} when i was in congress we spent a lot of time getting to know one another we spent a lot more time in congress than we do today he said i want to tell you a story one time i told my daughter i want you to go get a job and this was before he was majority leader i want you to go and get a job and i do not want you to use my name whatever you do do not use my name she came home that evening and she said daddy i found a job he was like oh great where did you find a job she said i got a job in the minority leader s office speaker wright a great storyteller that he was he said i just exploded and i said what you got a job at the minority leader s office did you tell them who i was she said dad you told me not to use your name he said that he immediately picked up the phone he called gerald ford up and he said gerald i need to apologize to you i want you to know that my daughter has accepted a job in your office and she is to report to your office first thing in the morning and apologize and say that she cannot accept the job he said that gerald ford said to him jim if your daughter wants to work here it won t be any problem at all he said marc can you imagine that happening today it really stopped and gave me pause just about how much things have really really changed speaker wright was an amazing person a person of great wisdom intelligence humility he would talk about how he lost the senate race and it was fine for him to lose that special election for the u s senate because things ended up working out for him in the u s house of representatives he could actually find humor even in something that was a big defeat for him i just wanted to thank him and i am so thankful that our paths crossed and that he was such an influence to me and so many others i can tell you that the city that i am from fort worth texas that the city is the great city that it is today because of the work and the statesmanship of jim wright his legacy continues to live on through so many others that continue to serve in congress today that are in other positions in office and in business mr speaker i am just very very grateful and very blessed that i knew speaker james claude jim wright mr speaker i yield back the balance of my time
\vspace{8mm}
i yield myself such time as i may consume mr speaker i rise today in strong support of h r 83 the omnibus appropriations bill for 2015 i want to commend the members of the house appropriations committee all of them especially the committee s distinguished chairman hal rogers for their hard work in writing a responsible proposal that will fund our national priorities and prevent a government shutdown i am also pleased that the bill includes critical reforms that will help our country avert a looming pension crisis today mr speaker roughly 10 million americans participate in a multiemployer pension plan men and women who have and continue to move our country forward builders truck drives carpenters electricians and store clerks to name a few these people worked hard and earned the promise that a pension would provide financial security in retirement yet for many that promise is now in jeopardy pension plans are on the brink of bankruptcy employers workers and retirees are stretched thin and a federal insurance agency is on the path to insolvency the multiemployer pension system is a ticking time bomb when the bomb goes off businesses will close their doors workers will be laid off taxpayers will be on the hook for a multibillion dollar bailout and retirees will have their benefits cut or wiped out entirely a crisis is staring us in the face and the {\bf \color{red} question} we have to answer is will we act will we do what is right and necessary to help fix this problem or will we simply kick the can down the road i believe we have a public duty and a moral responsibility to act my democratic colleague george miller and i have worked hard to craft a bipartisan legislative response to this looming disaster with the help of our friend dr phil roe and the work of many employers and union leaders we have reached agreement on a series of reforms that offer the best chance we have to protect taxpayers working families and retirees our bipartisan proposal includes tough medicine for a pension system in critical condition it requires higher premiums so the federal backstop can meet its obligations without taxpayer assistance it also provides new tools to trustees to help plans avoid insolvency including the ability to adjust benefits let me be clear if we reject this bill and continue the status quo benefits will be cut it is only a matter of time as plans go under the federal government inflicts maximum pain on the maximum number of people but if we offer trustees more flexibility they can avoid insolvency and provide retirees greater financial security we have a choice between an axe in the hand of a first year med student or a scalpel in the hand of a trusted surgeon this isn t easy no matter what happens retirees will face some difficult hardships that is why the proposal includes numerous protections but most importantly it ensures all retirees are better off than if we did nothing this isn t a perfect solution i am disappointed we couldn t do more to provide workers more options to plan for their retirement make no mistake this is the first step in addressing a tough problem and it won t be the last despite its shortcomings this is a strong proposal that deserves our support we cannot let this opportunity pass by this problem will be harder to solve after the bomb goes off i urge my colleagues to do what is in the best interest of workers employers and retirees by supporting this bipartisan agreement before i close mr speaker i want to thank some members of the staff who worked day and night to make this happen starting with my staff director juliane sullivan and workforce policy director ed gilroy i also want to thank brian kennedy megan o reilly and julia krahe of mr miller s staff for all of their hard work last but certainly not least i would like to offer my deep appreciation to a trusted member of my team andy banducci andy has poured more time and energy into this effort than anyone else and he has earned the right to a good night s sleep finally i would like to extend my sincere thanks to my colleague george miller who will leave this chamber after 40 years of public service without his courage and determination to do what is right this effort would not have been possible through it all he has been a trusted friend and ally george has long been a tireless advocate for working families from the start of his distinguished career down to these final moments in congress he will leave behind a lasting mark on the house and the education and workforce committee we haven t agreed on every issue but in the fine tradition of our committee we have always found a way to disagree without being disagreeable i have no doubt he will remain a powerful voice for students teachers and working families george thank you for your service and your friendship i wish you and your wife cynthia and family all the best mr speaker i reserve the balance of my time
\vspace{8mm}
i thank my friend the gentlewoman from ohio for being here this evening to have this special order to speak on an issue of such magnitude of importance to we the people of the united states important to the world that we never forget what took place the horror the utter destruction of humankind during the holocaust but in particular the focus of that destruction upon the jewish race it is important because we are seeing a rise quite frankly of anti semitism not only around the world but right here in the united states it takes different forms in different places but in the end has the same result of targeting and hurting one of the historically most vulnerable groups in our world the jewish people one of the things that has been the most concerning to me is the minimizing of the suffering of the jewish people during the holocaust frankly it is really outright disturbing i don t know if that does it justice that the white house of the united states of america the home of our president our present administration representing the same country that defeated nazi germany the same country that bore the greatest generation the same country that led the fight against anti semitism worldwide while recognizing from time to time it had to douse it here in the united states our country this same white house that i referred to deliberately refused to mention that the holocaust was designed to eliminate the jewish people from the face of the earth not a single mention of the final solution the final solution was to obliterate eliminate the jewish people off the face of the earth yes many people died in the holocaust as the gentlewoman made reference to so eloquently disturbingly but eloquently of the tens of millions of people who died we know of them historically but no race or religion was designated for elimination like the jewish people were the final solution was about ridding the jewish people from the face of the earth it is that simple it is imperative that this moment does not pass without some clarity what is clear is that the white house purposely removed the reference they are proud of it they doubled down they tripled down they removed the reference to the jewish people in its statement on international holocaust remembrance day why does this matter you may ask well first and foremost it feeds the extremists we know they exist let s face it extremists have welcomed this white house statement they love it they glorify it not just theoretically but literally literally white supremacists have welcomed the white house decision to leave any mention of the jewish people out of the holocaust remembrance secondly it matters because a lot of people in the world today either don t know that the holocaust happened or don t believe that the holocaust happened not just theoretically but literally don t believe that the holocaust took place literally a 2014 global survey of anti semitic attitudes found that 35 percent of people around the world have never heard about the holocaust maybe you can understand that but an additional 32 percent more importantly believe it is a myth or greatly exaggerated thirdly it matters because there are many holocaust survivors i know them and their descendents in the united states and throughout the world the actions behind the statement were just downright cruel and inhumane to them not just theoretically but literally cruel and inhumane literally groups that are dedicated to this issue are deeply deeply disturbed the anne frank center and others have raised their voices this is not just coming from democrats i don t want to mislead here at all there are a range of republican leaders and there are four of them and republican groups that have expressed their anger at the white house position on the holocaust but one entity we will come back to this house in a moment the white house hasn t seemed to have heard their outcry but the gentlewoman from ohio and i the democratic caucus we have heard what has taken place is wrong simply wrong you would think that the president would correct the situation in fact today he had the opportunity to condemn anti semitism at his press conference with prime minister netanyahu and he didn t do it in fact when he was asked on this very subject of the failure to mention the jewish people of the holocaust he used the opportunity not to clarify his position but to make reference to how great his election victory was in watching that press conference as disturbed as i was about the answer from our president i was more than a bit disappointed quite frankly by prime minister netanyahu s failure to challenge the president on that i wish prime minister netanyahu would have asked president trump to change his statement not to whitewash what was done but to change his statement on the holocaust i still hope that the prime minister does that in the time that he is here so this special order tonight will help us set the record straight not just on behalf of the millions of jewish americans across this country but to send a clear message to all those who engage in this type of behavior i ask this {\bf \color{red} question} where are our republican colleagues on this issue do you hear that silence we have given them opportunity after opportunity to speak out against what the white house has done but our republican colleagues refuse to criticize the white house for the omission of the jewish people in the holocaust resolution could you imagine for a moment what the outcry would have been had president obama accidentally omitted this putting aside purposely omitting it but the outcry if he had accidentally omitted the mentioning of the jewish people in his annual statement he never did that though nor did president bush as the gentlewoman from ohio ms kaptur has pointed out this was no mistake it was a willful omission yet still nothing from our republican colleagues the gentlewoman from ohio knows that i have offered a resolution we will continue to ask our republican colleagues to join us on that resolution asking the white house to set the record straight and to include the mentioning of the final solution and the attempt by the nazi regime to eliminate the jewish people from the face of the earth we will use every legislative mechanism possible to do that whether it is a motion to discharge whatever that will be i am putting my republican colleagues on notice because they must raise their voices they must raise their voices to what has taken place in this white house whether it is steve bannon and those who work within the cellar the deep cellar of the white house who came up with this resolution to purposely omit the mentioning of the jewish people our republican colleagues will either have to answer for the white house and defend it or condemn it you can t have it both ways i want to thank the gentlewoman from ohio once again for bringing us together it needed to be done we will continue to raise this {\bf \color{red} question} until the white house comes to its senses and sets the record straight and does no longer continue to enable holocaust deniers i thank the gentlewoman from ohio for holding this special order this evening\pagebreak

\section*{h}
mr chairman over the past couple of months i have worked tirelessly to find a reasonable compromise with the sponsor of {\bf \color{red} h} r 702 my friend and colleague from texas mr barton that would allow me and others with similarly situated constituencies to support this bill although i share the concerns of many of my democratic colleagues on how this bill might impact the environment and climate change i have always stated that i believe in the all of the above energy approach that balances environmental concerns and economic opportunities mr chairman yesterday in the rules committee i advocated for an open rule process that would have allowed democrats to offer amendments that would reflect priorities and concerns of the minority party in fact mr chairman i myself submitted an amendment that would have expanded access for minority and women owned firms to more fully participate in the energy supply chain which we know will be greatly enhanced if the export ban is lifted mr chairman although my friends in the environmental community wouldn t agree in my district we say oil is not just a commodity oil is indeed an economic opportunity mr chairman my most pressing concern is making sure up front and from the beginning that minority firms would be part of the pipeline economy and would directly benefit from vendor and contracting opportunities that lifting this ban would create instead mr chairman despite positive rhetoric from members of the majority party a closed rule was adopted while my comprehensive amendment was not allowed members are asked to vote now on trojan horse amendments that would do nothing to actually benefit minorities and women as my far reaching amendment was designed to do rather than shielding the majority party from charges of creating a multibillion dollar boondoggle for the energy industry today there is not much in this bill as currently drafted that i can point to as really benefiting all segments of the american population as i have said time and time and time again cut us in or cut it out cut us in or cut it out cut women in or cut it out cut the minorities in or cut it out
\vspace{8mm}
i ask unanimous consent that all members may have 5 legislative days in which to revise and extend their remarks and include extraneous material on {\bf \color{red} h} r 2742 currently under consideration
\vspace{8mm}
mr chairman i yield myself the balance of my time have you ever had a moment at which you are approaching the microphone and you have got to accept that we are all passionate about our views and you have heard some things that shall we say start to get your blood pressure moving a bit but let me see if i can do this without being hyperbolic and then walk through some of the realities of the information that is laid out in front of me right here first i do want to respond to something that ranking member johnson said i want to first caveat that she has always been very kind to me but we have the confirmation from the epa itself and we will put the documents into the record that they are perfectly capable of blinding anything that is confidential anything that is personal i mean we have the comments from administrator mccarthy on march 7 walking us through that they can do this and they didn t see it as a real problem let me walk through something else that i am finding sort of absurd and i am having a little trouble finding the best way to articulate this we spent about an hour in our office sort of just searching the internet on this subject if you go back about a decade ago a number of our friends on the left were demanding something almost identical to this so what is different it wouldn t happen to be a different philosophy a different president a different party in the white house would it let me back up and say why do i embrace this secret science bill {\bf \color{red} h} r 4012 i genuinely in every fiber of my being believe that we will get better policy better design more creative ideas because whether you are on the left the right or are just an active addition you do not know whether the epa rule sets are optimal you may believe they are but we are doing it on faith peer review is wonderful except for the fact that the peer reviewers don t see the underlying data the beauty of this piece of legislation is that neither you nor i right now knows in the absolute collective analysis whether the epa is even going far enough or whether it is going too far or whether there is another approach that would be dramatically more efficient what happens when that researcher gets his hands on a linear data set and matches it up with something else that no one had thought of putting in there and all of a sudden discovers the noise in the data that there are opportunities to do it better faster more efficiently to save lives or to maybe even do it cheaper you will not know that until the cabal that right now has the franchise on the information on the brokerage of the data is broken up what is so stunningly disheartening here is that much of this concept if you go back and look at the speeches from the president in 2007 and 2008 and at memos from the president 18 months ago from omb demanding this saying this was the wave of the future if you embrace science but not the science of an elite few the fact of the matter is our nation our country and our world is made up of really smart people who have the right and the ability to give us input to do this better i beg of my fellow members here to stop being afraid of true transparency stop defending the incumbent class that thinks it has the only legitimate scientists who have the right to put forward what our future looks like i may be behind this microphone in a couple of years from now if this bill passes saying i never knew we weren t going far enough you may be behind that microphone over there saying the crowd analysis of the data says there was a dramatically better way but we need to pass this bill to have that opportunity mr chairman i yield back the balance of my time the administrator of the environmental protection agency washington dc march 7 2014 hon lamar smith chairman committee on science space and technology house of representatives washington dc dear mr chairman thank you for your letter of february 14 2014 regarding the united states environmental protection agency s epa s response to a subpoena duces tecum subpoena from the committee on science space and technology committee as you note in your letter during and immediately after my november 14 2013 appearance before your committee we agreed to additional dialogue regarding the epa s response to the subpoena i understand that our staffs have had several discussions since that date and made significant progress toward a common understanding of this matter i want to thank you and your staff for your willingness to engage in these discussions as i believe they have been both productive and constructive your subpoena sought data from the american cancer society and harvard six cities cohorts as well as analyses and re analyses of that data in particular the subpoena sought data from studies that utilized data from the american cancer society and harvard six cities cohorts once the epa received the subpoena we conducted a diligent search for data as well as analyses and re analyses of that data that were already in our possession custody or control that would be responsive to the subpoena in addition we considered what data as well as analyses and re analyses of that data were not in our possession custody or control on the date we received the subpoena but that may still be within the scope of the committee s subpoena for data as well as analyses and re analyses of that data that were not in the epa s possession custody or control but that could still be considered within the scope of the subpoena the epa sought to identify a legal authority for the agency to obtain that information so that it could be provided to the committee in this case the shelby amendment public law 105 277 provides the epa with the authority to obtain certain research data that was not in the agency s possession custody or control on the date we received the subpoena and the epa utilized that authority to obtain that data the actions taken in response to the subpoena are detailed in an enclosure enclosure 1 to this letter and included multiple interactions with the third party owners of the research data in an effort to obtain that data once the agency successfully obtained the research data we undertook a review of this data to determine whether the release of the data would raise privacy concerns the agency sought the assistance of the centers for disease control in this inquiry as well in an effort to ensure the privacy of the subjects of the data was not compromised through its efforts the epa located within its possession custody or control or obtained through its authority the data for five studies listed in the subpoena any other data as well as analyses and re analyses of that data that may be within the scope of the subpoena whether specifically listed in the subpoena or not are not and were not in the possession custody or control of the epa nor are they within the authority to obtain data that the agency identified however the issuance of the subpoena does not provide the agency with any additional authority to obtain data as well as analyses and re analyses of that data that we otherwise do not have the authority to obtain all responsive data as well as analyses and re analyses of that data located or obtained during our efforts to respond to the subpoena have been provided to the committee the epa provided that data to the committee through letters sent prior to our receipt of the subpoena and then our letters responding to the subpoena of august 19 2013 september 16 2013 and september 30 2013 the epa provided the committee with the data for these five studies in exactly the same format the data were provided to us importantly the agency was able to work through the various privacy concerns so that we would not need to de identify any of the data as of the epa s letter of september 30 2013 the agency has provided the committee with all of the data covered by the subpoena that the agency has obtained or has the authority to obtain under the shelby amendment additionally the epa has not withheld any data in our possession that is responsive to the subpoena thus the epa has completed its response to the subpoena the epa acknowledges however that the data provided are not sufficient in themselves to replicate the analyses in the epidemiological studies nor would they allow for the one to one mapping of each pollutant and ecological variable to each subject for the reasons explained in our previous letters on this topic these acknowledgements do not call into question the epa s reliance on these studies for regulatory actions your february 14 2014 letter also requests the grant agreements related to the studies covered by the subpoena and those documents are being provided with this letter these epa grant agreements span from 1998 to 2006 and contain a variety of data access provisions despite that variation the epa has reviewed each of the agreements and determined that each grant agreement contained data access provisions that are consistent with the epa grant regulations at the time of the award the epa s current practice is to incorporate into our grant agreements a reference to the agency s regulations regarding access to research data funded by the grant thank you again for the opportunity to explain the actions the epa took in responding to your subpoena sincerely gina mccarthy\pagebreak

\section*{bill}
it is time for my republican colleagues to do more than repeal it is time for my republican colleagues to pass a {\bf \color{red} bill} who can deny that the house is controlled by my republican colleagues they control every committee they control every subcommittee they are in control who can deny that they have the opportunity to pass the perfect {\bf \color{red} bill} to deal with health care in this country where is the {\bf \color{red} bill} that will deal with closing the doughnut hole for senior citizens where is the {\bf \color{red} bill} that will help those who have preexisting conditions to acquire insurance where is the {\bf \color{red} bill} that will deal with the cap that has been placed on insurance prior to obamacare they are in charge the logical question is why haven t they passed a {\bf \color{red} bill} since the affordable care act passed more than 3 years ago it is time to do more than repeal you have to have a {\bf \color{red} bill} it is time for my republican colleagues to do the logical thing to do the judicious thing to do the prudent thing pass your {\bf \color{red} bill} then we can see how obamacare passed to what you have which of course is the perfect {\bf \color{red} bill} it s time to do more than repeal it is time to pass a {\bf \color{red} bill}
\vspace{8mm}
mr chairman the underlying language in this {\bf \color{red} bill} was a bipartisan agreement that was worked out in the last congress when we were marking up this {\bf \color{red} bill} previously it was brought up that minority representation would be important to this {\bf \color{red} bill} and so the chairman of the committee mr hensarling actually stopped the deliberation there and worked in a bipartisan way across the aisle with ms waters to make sure that we put language in the {\bf \color{red} bill} that would encourage the director to make sure that women and minorities business concerns on the small business advisory board were taken into consideration we have addressed that and we kept that language that was agreed to and by the way was passed by a voice vote mr pittenger accepted that amendment and the {\bf \color{red} bill} reported out of the committee 53 5 so basically we have kept our word and kept in the spirit of the agreement that was negotiated in the previous congress and that language is in this underlying {\bf \color{red} bill} i would encourage folks not to vote for this amendment i reserve the balance of my time
\vspace{8mm}
what is really happening here is the most extreme anti immigrant voices in the republican party using the crisis as a political cover to repeal a commonsense policy like the dream act and the speaker has caved once again to those voices representative steve king described the underlying legislation as something that he could have ordered off the menu furthermore the rules are of course closed setting the record anew for the most closed rules in any congress this {\bf \color{red} bill} does stop short of catapulting those children into mexico and then leaving them to walk to their home countries but it certainly doesn t do very much since the discussion in the house of representatives for several years now has been what to do about immigration it really is a sorry path that we have reached the condition we are in right now a one house {\bf \color{red} bill} a senate that is gone and a president who won t sign it if we learned anything this week we learned from speaker boehner s comments on his blog that the president should do more not less contrary to the reason why they sued him and we do hope that the president will do that and bring a more humane solution to this as almost all religions in the united states have asked us to do i reserve the balance of my time\pagebreak

\section*{days}
i thank the gentlewoman now why would an oregonian insert himself into the perpetual water wars in california well first off this bill has had no hearings as you can see from the debate here on the floor there is extraordinary disagreement over the potential impacts of this legislation that is not just critical to californians it is critical to oregonians i have a letter here from the pacific fishery management council they believe that this could have a hugely detrimental impact on some audit species which compose about 80 percent of the california fishery and about 50 percent of the fishery in oregon we went through this before about a decade ago where there were inadequate outflows there were problems with the forge fish the smelt and the returning salmon and we had a season that was closed for 2 years it put many many oregonians out of work there was impact beyond commercial fisheries and coastal communities on recreational fisheries it cost us hundreds of millions of dollars we got a couple of hundred million dollars in federal relief the experts the pacific fishery management council and their lawyers who have read this bill believe it does change the management of the water in ways that are detrimental and would void the biological opinion and would probably put us back into another couple of no fishing years a few years down the road given the cycle of salmon particularly section 103 d 2 and section 103 c i have heard here on the floor despite the fact no hearing has been held the bill just burbled up very recently that on one side they are saying no don t worry it will not have a detrimental environmental impact and if it does well we will stop doing it but i just looked at that section of the bill and it doesn t quite say that definitively in fact it changes the standards and then it says if additional negative impacts might happen then the secretary could suspend some of the provisions of this bill not exactly certainty and we need some certainty here for our fisheries we have been hurting for years last year we had a good year thankfully we are still dealing with buybacks because of reducing the size of the fleets from past problems fishermen are burdened with the buyback year in year out i just got the terms of that adjusted in the ndaa they had a payday loan from the federal government now we got them a reasonable loan from the federal government the government didn t even pay for their buyback heck in the northeast they paid for a couple of buybacks no we had to pay for our own with a payday loan now we are going to jeopardize the fleet 1 2 or 3 years out because we won t have the returns with the endangered species so this is a bad idea to do in the waning {\bf \color{red} days} of a congress to bring forward a bill which is controversial over which there is disagreement on the actual language in the provisions of the bill and which my experts the pacific fishery management council say would be detrimental and would cause those problems pacific fishery management council portland or december 6 2014 hon jared huffman u s house of representatives washington dc dear mr huffman thank you for your letter of november 17 and follow up on december 3 requesting pacific fishery management council pacific council comment on legislation related to operation of the state water project and central valley project in california hr 5781 and its potential impacts to fisheries although the timing of the bill did not allow for full council deliberation we present the following concerns which are consistent with previous comments the council has made on similar legislation absent changes in the legislation to address these concerns the pacific council does not support hr 5781 moving forward hr 5781 would override endangered species act protections for salmon steelhead and other species in the bay delta in order to allow increased pumping from the delta in excess of scientifically justified levels these measures also protect salmon stocks not currently listed under the esa which are a primary source of healthy sport and commercial fisheries from central california to northern oregon the bill introduces a new standard for implementing the endangered species act concerning central valley salmon and delta smelt a keystone species in the bay delta ecosystem see sec 101 3 and 102 b 2 a it is unclear how severe the negative effects of this new standard might be but it would certainly impact current water management policy that protects esa listed salmon stocks from further decline and helps prevent currently healthy stocks from becoming listed under the esa the bill contains several provisions that override the salmon and delta smelt biological opinions for example section 103 d 2 section 103 c and others section 103 could result in dramatically higher pumping than is authorized under the biological opinions and would cause significant harm to migrating salmon and steelhead and other native species the 1 1 inflow to export ratio for the san joaquin at vernalis overrides the reasonable and prudent alternatives to standard operations that were set out in the 2009 central valley biological opinion in order to protect sacramento river winter run chinook and other salmonid species further degradation of salmon habitat is contrary to the provisions of the magnuson stevens act sec 305 b 1 d and something the pacific council strongly opposes section 103 f 2 provides exemptions for mitigation of negative effects on listed fish species which alleviates the project from compensating fisheries for negative effects of its operations it is unclear if there is an exemption for mitigation of negative effect on non listed salmon stocks exempting mitigation responsibility for harm to salmon populations provides the exact opposite incentive to the kind of salmon protection and enhancement advocated by the council and essentially amounts to redistributing the value of salmon fisheries to agricultural or municipal interests as well as increasing the risk to esa listed fish stocks threatened with extinction additionally the pacific council is concerned about whether central valley projects are achieving their current mitigation responsibility and providing these exemptions could preclude seeking remedy if this bill moves forward it should provide direct mitigation for the proposed actions and risks to which it would subject fish populations and fishing communities not avoiding this appropriate responsibility in 2008 and 2009 158 million in congressional aid was provided to deal with the disaster of the closure of ocean salmon fisheries off california and oregon south of cape falcon due to a collapse of the sacramento river salmon stocks these fisheries are an important source of jobs for coastal communities which cannot be replaced simply through disaster relief without adjustments to this bill we fear such a disaster could be repeated in the reasonably near future thank you again for the opportunity to comment on this legislation please don t hesitate to contact me or ms jennifer gilden of the pacific council office if you have any further questions sincerely d o mcisaac ph d executive director
\vspace{8mm}
i ask unanimous consent that all members have 5 legislative {\bf \color{red} days} to revise and extend their remarks on the subject of this special order
\vspace{8mm}
i thank the gentleman for yielding and thank him for giving us this opportunity to discuss an important matter the integrity of congress on the floor of the house i too want to join the distinguished majority leader mr cantor in praising the leadership of congresswoman louise slaughter our ranking member on the rules committee and congressman tim walz for their extraordinary leadership over time their persistence the approach that they have taken to this to remove all doubt in the public s mind if that is possible that we are here to do the people s business and not to benefit personally from it i listened attentively to the distinguished majority leader mr cantor s remarks about the stock act and its importance and it just raises a question to me as to if it is so important and it certainly is why we could not have worked in a more bipartisan fashion either to accept the senate bill which was developed in a bipartisan fashion and passed the senate what was it 94 6 it s hard to get a result like 94 6 in congress these {\bf \color{red} days} but they were able to get the result because they worked together to develop their legislation we had two good options one was to accept the senate bill or to take up the slaughter walz legislation which has nearly 300 cosponsors almost 100 republicans cosponsored the original stock act the discharge petition has been calling upon the leadership to bring that bill to the floor what s important about that is that if we passed that bill we could go to conference and take the best and strongest of both bills to get the job done instead secretly the republicans brought a much diminished bill to the floor it has some good features so i urge our colleagues to vote for it to bring the process along what s wrong with it though is that it makes serious omissions and i want to associate myself with the remarks that had been made earlier but i think they bear repetition in any event senator grassley s remarks are stunning it is really a stunning indictment of the house republicans in terms of their action on this bill and i know my colleague has read this into the record already but i will too senator grassley said it s astonishing and extremely disappointing that the house would fulfill wall street s wishes by killing this provision that would be the provision on political intelligence the senate clearly voted to try to shed light on an industry that s behind the scenes if the senate language is too broad as opponents say why not propose a solution instead of scrapping the provision altogether i hope to see a vehicle for meaningful transparency through a house senate conference or other means if congress delays action the political intelligence industry will stay in the shadows just the way wall street likes it well the senator s statement is very widely covered the hill today has a big full page grassley republicans caved iowa senator says house doing wall street s bidding i think it is important to note that on the senate side there was interest in doing this study that is now in the house bill and it was rejected by the senate by a 60 39 vote to include the political intelligence provision in the bill rejecting the study now that that has already been rejected in the senate it s resurrected on the house side a weakening of the bill so whether it s the political intelligence piece proposed by senator grassley or senator leahy s piece about corruption i think it is really important that those two elements be included in the bill a good way to do that to find a path to bipartisanship in the strongest possible bill is to pass the bill today despite its serious shortcomings and it is hard to understand why the shortcomings are there but nonetheless they are but pass the bill today and go to conference to pass earlier or to accept the senate bill or to take the original stock act strong stock act to the floor both of those were rejected pass this bill and go to conference it is very important that the house and the senate meet to discuss these very important issues with all due respect to a study on political intelligence that s really just a dodge that is just a way to say we re not going to do the political intelligence piece so again with serious reservations about the bill but thinking that the better course of action is to pass it and i don t want anybody to interpret the strong vote for it to be a seal of approval of what it is but just a way of pushing the process down the line so that we can move expeditiously to go to conference for the strongest possible bill i want to close again by saluting congresswoman louise slaughter and congressman tim walz for their relentless persistence and dedication to this issue had they not had this discharge petition and the nearly 300 cosponsors bipartisan nearly 100 of them republicans i doubt that we would even be taking up this bill today so congratulations and thank you\pagebreak

\section*{r}
i ask unanimous consent that all members have 5 legislative days in which to revise and extend their remarks and to include extraneous material on h {\bf \color{red} r} 807
\vspace{8mm}
i thank the gentleman for yielding mr speaker i rise today in very strong support of h {\bf \color{red} r} 1961 legislation that my colleagues and i introduced to save the delta queen steamboat and i want to particularly thank the gentleman from missouri st louis in particular my democratic colleague lacy clay for his leadership on this particular issue this legislation is basically one line it doesn t cost a penny and it has two very important functions it preserves an important piece of american history and it supports american jobs mr speaker h {\bf \color{red} r} 1961 reinstates the delta queen s grandfathered status not an earmark the grandfathered status from a law that prohibits wooden boats which the superstructure of the delta queen is the hull of it is steel for carrying overnight passengers the delta queen is actually capable of carrying up to 176 passengers comfortably overnight and under the law as it currently exists 50 is the cutoff point congress granted the delta queen a reprieve from this law for the last 40 years so for 40 years the united states congress granted this exemption it did so because she was constructed before the law was in place and because the law was intended for boats at sea not riverboats boats oceangoing vessels at sea it was never intended for river faring boats like the delta queen that s why the congress granted this exemption for 40 years the queen s grandfathered status was uninterrupted for 40 years until management concerns stalled the continuation back in 2008 since congress revoked its ability to operate the boat has been chained to a dock discord and disagreement won that day but today hopefully it will be different today we have a renewed coalition of support democrats and republicans have worked together on this issue it passed by voice vote with no votes against it in the transportation committee and maybe most importantly the boat s new management and union are working together to return this vessel and the jobs she provides to full operation so this is a situation where management and the union are not fighting they may have been back in 2008 they re not now they re together on this they re both requesting that we pass this particular legislation today so that the delta queen can once again ply the rivers the mississippi the ohio and bring jobs to communities all up and down those rivers with all the gridlock in washington this bill is a welcome show of bipartisanship for a change i wish we had more of that around this place but this really is a bipartisan bill it s supported by the seafarers international union by the american maritime officers and by the national trust for historic preservation for example it s cosponsored by a diverse list of republicans and democrats including the entire ohio delegation including my colleague and i want to thank him for his leadership on this issue brad wenstrup from the second district right next to my district the first district in the greater cincinnati area he has been a leader on this as has congressman massie across the river and as i mentioned before congressman lacy clay from missouri and many other members it also has the support of transportation committee chairman shuster on the republican side and ranking member rahall and i would like to read a quote from the gentleman from west virginia mr rahall the ranking member who was unable to be here today actually i think he is driving here and will be here for votes but couldn t make the debate but this is what he said back in the transportation committee itself and i am quoting here from his testimony i m in favor of preserving an icon of our american heritage the delta queen in light of the support that this bill has from the seafarers the seafarers union and the fact that this means good paying jobs and that a unique part of americana would be restored to service i support the pending legislation that s the bill that we are dealing with here today and in the past this effort was even cosponsored by two men who rarely see eye to eye senator mitch mcconnell and then senator barack obama both of them supported this back in 2008 i owe thanks to every lawmaker who cosponsored this measure and i owe a special thanks as i mentioned to the gentleman from st louis missouri mr clay without whose help this wouldn t be possible today to my colleagues who have raised issues about the vessel s safety i hear you safety must always be a top priority so let s discuss it for a minute this vessel is equipped with a fully automated environmental detection system that uses over 300 sensors to detect heat smoke and co2 for example it also has a state of the art sprinkler system a coast guard trained and certified firefighting crew and round the clock watchmen patrolling the vessel 24 hours a day it should also be noted that the original legislation from 1965 and i mentioned this before was intended for oceangoing vessels that s why it was called the safety at seas act not the safety on the rivers act as a river vessel the delta queen is never more than a mile from shore and can be landed and evacuated in minutes if need be fortunately that s never been necessary with the delta queen in its 80 years basically in traveling and 60 years on the rivers of the mississippi and ohio so oceangoing vessels we are talking about vessels that oftentimes are hundreds of miles perhaps even over 1 000 miles from land in this case we re talking about never more than one mile that s why the delta queen is different it was the only river vessel that this really applied to because of its size and the fact that it could take more than 50 passengers that was the problem and to clear any misunderstanding the legislation does not relieve the boat managers of their responsibility to deal with safety issues in order to obtain a certificate of inspection a coi from the coast guard the vessel will have to address united states coast guard concerns the managers already have a detailed list of things they know will need to be upgraded which include replacing the vessel s boilers in all likelihood and steam lines with modern fully automated welded construction boilers and steam lines so the issues that were concerns back in 2008 which my distinguished colleague mentioned before these are all going to be taken care of and should be otherwise we wouldn t be supportive of this bill this bill does not issue a green light this bill unlocks the private resources necessary to make this multi million dollar restoration effort possible at the end of the day if the boat doesn t satisfy the coast guard they don t get a coi and they don t sail they don t paddle they don t move they don t travel at all while objections on the grounds of safety are reasonable i feel that safety may be a convenient argument really not a justified argument let me close at this point by saying that the delta queen is beloved by many particularly many cincinnatians who spent years watching her sail into our city to unload passengers at dawn and head out back with a new group of people at dusk i think many of us would like to give her that opportunity up and down the mississippi and the ohio again it means jobs for many people in many of these communities i ask my colleagues to join us in supporting this bill for two principal reasons jobs and american history members can support this by voting in favor of h {\bf \color{red} r} 1961
\vspace{8mm}
let me begin by thanking chairman mccaul and ranking member thompson for their dedication to improving our cybersecurity posture since the chairman and i founded the congressional cybersecurity caucus together nearly a decade ago i have come to firmly believe that cybersecurity is the national and economic security challenge of the 21st century and both congress and the executive branch must take steps to recognize and mitigate the risks we face in our hyper connected society thanks to the leadership of the chairman and ranking member the committee on homeland security has consistently been at the forefront on these issues and while much remains to be done we are worlds away from when i originally took the cyber subcommittee gavel in 2007 the bill we have before us today is a testament to those efforts and i strongly support this latest iteration of cisa to reorganize the national protection and programs directorate nppd and enhance the capabilities and the profile of dhs s cybersecurity activities as one of its core missions dhs is charged with helping federal agencies and critical infrastructure owners and operators secure themselves against physical and cyber attacks for the past decade that mission has been carried out by nppd a small headquarters component of the department since its establishment nppd s role in defending the nation and the gov domain from cyber intrusions has grown in concert with the increasing threat to our networks it s a growth that the committee and the congress as a whole has recognized and encouraged with the passage of laws including the national cybersecurity protection act of 2014 which authorized the national cybersecurity and communications integration center and the cybersecurity act of 2015 which made nppd the federal government s primary hub for cyber threat indicator sharing today nppd is home to two of the premiere computer security incident response teams in the world and has been recognized as the whole of government asset response lead in the national cyber incident response plan it also continues to lead efforts in protecting federal networks through the federal network resilience division which assists other agencies with risk management guides enterprise security policy and implements programs like continuous diagnostics and mitigation and einstein nppd is clearly acting in an operational capacity today but despite this fact congress has not yet elevated nppd s standing to be commensurate with these added responsibilities h {\bf \color{red} r} 3359 acknowledges the evolution of the component by transforming nppd into an operational agency on par with tsa or customs and border protection as part of the reorganization nppd will be renamed the cybersecurity and infrastructure security agency or cisa to accurately reflect its role this restructuring was the top legislative priority of dhs secretary jeh johnson before he left office and i am grateful that secretary kelly took up the mantle in the new administration bringing clarity with the new agency structure also stands to benefit the many cyber defenders working tirelessly at the department to keep us safe i have often said that all of the risk mitigation policies and intrusion detection systems in the world are nothing without a skilled workforce congress and the department have been working jointly to reduce the shortage of cybersecurity analysts at nppd and it is my hope that an empowered cybersecurity and infrastructure security agency will be able to compete for the best cyber talent after all what mission is more exciting than protecting your fellow americans from the canniest of adversaries attempting to do us harm in this new domain i hope that all of the young people considering a career in this emerging field young people like the brilliant cybercorps students i enjoy speaking with will look at congress s support for dhs s cybersecurity work and jump at the opportunity to be in the vanguard at this new agency mr speaker i also want to speak about the important clarity h {\bf \color{red} r} 3359 brings to a broader policy debate that has been kicking around washington dc for some time now i serve on the house armed services committee where i am privileged to act as ranking member of the subcommittee on emerging threats and capabilities in this role i oversee united states cyber command and i have the utmost respect for the service members in uniform defending our country in the digital domain i have also had the privilege to serve on the permanent select committee on intelligence where i heard weekly about the all too often unsung heroes of our intelligence community and their efforts to protect our national interests in cyberspace i say this mr speaker because i want to be clear that i have a deep understanding of and appreciation for our military and intelligence services cybersecurity prowess but i also believe that the powers and authorities of those entities are rightly constrained when it comes to domestic activities protecting our domestic cyber assets in peacetime needs to be the responsibility of a civilian organization and that organization is the cybersecurity and infrastructure security agency created under this bill we saw this debate play out during consideration of the cybersecurity information sharing act of 2015 where it was also decided in favor of a civilian hub the nccic that is at the heart of nppd i hope passage of h {\bf \color{red} r} 3359 will help move the debate on from where authorities should be housed and instead focus on the operationalization of said authorities mr speaker as i mentioned at the outset this bill owes its existence to the collaborative efforts of chairman mccaul and ranking member thompson but it also reflects the bipartisan spirit of two of my good friends who head the subcommittee on cybersecurity and infrastructure protection mr ratcliffe and mr richmond and like any effort of this body it owes a great deal to the staff who work tirelessly behind the scenes supporting our efforts in particular i would like to kirsten duncan and moira bergin the majority and minority staff directors for the cyber subcommittee for helping to get this bill to the floor and i would also like to thank their predecessors brett dewitt and chris schepis for laying the groundwork for its consideration this congress this bill is important it s bipartisan and it s overdue i hope my colleagues will join me in supporting this important measure and i hope the senate moves swiftly to pass it through their chamber\pagebreak

\section*{unanimous}
i ask {\bf \color{red} unanimous} consent to insert my statement in the record that the house should focus on the real priorities of americans instead of another attack on women s health care
\vspace{8mm}
i yield to the gentleman from new york mr engel for a {\bf \color{red} unanimous} consent request
\vspace{8mm}
i ask {\bf \color{red} unanimous} consent that all members may have 5 legislative days to revise and extend their remarks\pagebreak

\section*{act}
mr chairman this is a simple good government provision it says that when a contractor goes over budget or is behind schedule the contractor should not be rewarded for that none of the funds made available in this {\bf \color{red} act} may be used to pay for bonus awards to contractors who work on projects that are behind schedule or over budget the provision that we are talking about here appears in the senate transportation housing appropriations bill that was reported out of the committee in the senate last week it should appear in our bill and it should be signed into law nothing in this amendment places a blanket ban on bonuses to contractors what this amendment does however is to demonstrate that congress expects federal projects to be delivered on time and on budget we have heard so many words over the years in this chamber about waste fraud and abuse this simple amendment accurately cracks down on those examples of waste fraud and abuse that arise and prevents taxpayer money from being squandered if projects are not delivered on time and on budget this amendment simply ensures that bad contractors are not rewarded extra for that poor performance with regard to the terms that are used the term bonus award refers to the federal acquisition regulation title 48 of the code of federal regulations subpart 16 4 having to do with incentive contracts that term is defined in that provision with regard to the term work on projects that simply refers to the contractor s contract with regard to the term behind schedule that refers to the time of delivery that is a provision that is in every contract in far 52 211 8 or far 52 211 9 the regulations specifically provide for time of delivery with a delivery schedule and that is the term that is used in the regulation and also in the contract itself those provisions are proscribed in the federal acquisition regulations in 48 c f r subpart 11 4 specifically far 11 404 the term over budget is very simply a reference to the contract award itself the federal acquisition regulations proscribes a specific form for that purpose in 48 c f r 53 and that is standard form 33 in box 22 of standard form 33 is the contract award amount if the contractor goes over budget the contract has exceeded the amount that appears in far 52 3 of 33 in the award amount box in box 20 the provision refers to cost reimbursement awards and it refers to time and material awards if the goes over budget on a firm fixed price award the contractor bears that expense if the contractor goes over budget on a time and materials award or a cost reimbursement award and then seeks a bonus on top of that from the government then that is what we are prohibiting here these are terms that are well recognized in the world of federal contracting this provision accurately targets overpayment to contractors extra payment to contractors bonus payment to contractors when they have gone behind schedule or they are over budget i submit that the senate was wise to include this in its bill we should do the same i ask my colleagues respectfully for their support i yield back the balance of my time point of order
\vspace{8mm}
i thank the gentleman for yielding and for organizing tonight s special order on the affordable care {\bf \color{red} act} which is helping to make health care a reality for millions of americans across our nation luckily california is one of the states that has a plan it has bought into the affordable care {\bf \color{red} act} as a result thousands of california are now benefiting from what we in california call covered california which is the aca plan there by enrolling in the affordable care {\bf \color{red} act} parents and their children no longer have to endure illnesses or painful injuries because they can t afford a doctor parent don t have to worry about their children getting a preventable illness because they can t afford to have them vaccinated or treated for a chronic preventable disease why because under the aca many immunizations and preventative services are free seniors and adults are also eligible for free preventive services including annual checkups annual mammograms prostate cancer screenings and immunizations young adults including 435 000 young californians don t have to worry about being a burden on their family if they get sick or are in an accident because they can remain on their parents insurance until age 26 and get affordable insurance after that also critical is the fact that under the affordable care {\bf \color{red} act} no one can be denied health care coverage because of a preexisting condition the aca is a wonderful opportunity as you have pointed out for uninsured americans to get the health care that they need to improve the quality of life for themselves and for that of their family and i would like to just give one example of that a constituent of mine from the city of bell by the name of roberto rivas is in his mid twenties on december 21 2013 he arrived at 6 a m to enroll in a health insurance plan before going to work at kfc where he is not offered any health insurance he is also a full time student at trade tech studying chemistry he would like to use his education to study proteins and to research viruses such as hepatitis and other infectious diseases until the age of 21 along with his 10 year old sister he was covered by his mother under medi cal when he turned 21 he was no longer eligible for medi cal he lost that insurance and was left completely without any health insurance whatsoever shortly after he began suffering from breathing problems he went to a doctor and found out that he had pneumonia later after being treated for that pneumonia he received a medical bill for 4 663 he had no insurance to cover that he even asked for charity care services to help cover his expenses but was denied that request robert said as a minimum wage worker and a full time student it is hard to get health insurance thanks to obamacare now i can go to school and not stress about getting sick and ending up in the hospital i m calling everybody in my family to tell them i m enrolled in health care and that they need to come out and get covered too robert rivas was also astounded by the service the friendly faces and the applause he received when he enrolled and he says to know so many people actually care about me getting health insurance is great this is just one example of the millions of americans who are benefitting from what we call obamacare or the affordable care {\bf \color{red} act} i am hoping that more californians who have not applied and americans across the country who are uninsured and can benefit greatly by enrolling in health care that they don t miss out there are only 5 days left until the enrollment deadline of march 31 i hope that today they will visit healthcare gov or use any services which you have already outlined to enroll in the affordable care {\bf \color{red} act} for themselves and for their families
\vspace{8mm}
mr chair i thank chairman hensarling the gentleman from texas for his leadership on this issue and onso many regulatory reform issues that we will be addressing this week and in the future mr chair i am proud to sponsor and bring to the floor h r 78 the sec regulatory accountability {\bf \color{red} act} this legislationfits perfectly with the theme of the week here in the house to advance key regulatory reform ideas as a change of pace from theoutgoing administration for the past 8 years the amount of regulatory burden that has been placed on americans and small businesses has beencrushing in 2015 federal regulation cost almost 1 9 trillion that is nearly 15 000 per household in a hidden compliancetax the obama administration issued over 600 economically significant rules which are those that have an economic impact ofover 100 million as a result of this wave of regulations we have been part of the slowest economic recovery in our lifetimes we now have an opportunity to enact policy that ensures smart regulation going forward so that we are doing things in thebest and most efficient way the people have spoken mr chair business as usual in washington is over and it is time to dothings differently there is indeed a better way this legislation is really about what everyday americans do when they are making major life decisions in weighing the costsand the benefits the pros and the cons whether it is buying a car buying a home deciding whether to take out a loan to go toschool everyone must consider the core economic factors when making important life decisions the sec regulatory accountability {\bf \color{red} act} places statutory requirements on the sec when issuing rulemaking that ensures that first they identify the problem that regulation is trying to address second they weigh the cost and benefits to ensure thatthe benefits justify costs of compliance and thirdly they identify and assess whether there are any available alternatives torulemaking additionally this bill contains a provision that requires the sec to review its existing regulations every 5 years at theminimum to determine whether any such regulations are outdated ineffective or excessively burdensome as well as requiringthe sec to modify streamline repeal or even to expand regulations based on that review as a regulator of our capital markets the sec has an immeasurable influence on our economy and the ability of smallbusiness and entrepreneurs to be able to access capital in order to innovate grow and most of all create jobs i strongly believe that this legislation is nonpartisan and common sense and what our government regulators should have beendoing in the first place the american people deserve a break from the irresponsible regulation they have grown accustom to overthe past 8 years there is a better way i ask my colleagues to support this commonsense piece of legislation and urge passage of it through the house\pagebreak

\section*{order}
mr chairman i yield myself such time as i may consume mr chairman we are here today on the floor with the wrda bill we are back in regular {\bf \color{red} order} this bill reasserts congressional authority making sure that congress has its say on these matters this bill addresses specific federal responsibilities that strengthen our infrastructure and it is fiscally responsible if we pass the manager s amendment there are 31 chief s reports and 29 feasibility studies which touch all corners of the united states i know members on both sides of the aisle have projects in there that are extremely important to their district to their state and of course to the nation it certainly was my goal for this to come to the floor in a bipartisan manner just the way it came out of committee unfortunately it did violate a house rule and we had to strip a part of that bill out but i just want to say again as i opened i agree with mr defazio and you heard as he just pointed out there are many members on our side of the aisle that agree we have got to figure out a way to move this forward so that congress continues to have a say and that those dollars that people pay to use the ports they pay that fee and when it goes into that trust fund it is spent on its intended purpose it is just wrong it is absolutely wrong that we don t do that we are going to pass this bill on the floor here tomorrow i will continue to work with the ranking member to find a solution because it is my goal to be here next congress and to have another wrda bill on the floor and address this problem and continue to pass good legislation that strengthens our infrastructure and strengthens america s competitiveness in the world mr chairman i yield back the balance of my time
\vspace{8mm}
house resolution 770 provides for consideration of h r 5278 the puerto rico oversight management and economic stability act or promesa the resolution provides for a structured rule and makes in {\bf \color{red} order} eight amendments this bill addresses a very serious issue as it relates to the financial situation in puerto rico the government of puerto rico has racked up over 118 billion in debt they have already defaulted on portions of their debt in may and they face another deadline on july 1 the territory has reached a point where they can no longer meet the basic demands of their citizens the constitution makes clear that congress has the authority over territories article iv section 3 clause 2 of the constitution states the congress shall have power to dispose of and make all needful rules and regulations respecting the territory or other property belonging to the united states after hearing calls for greater autonomy in 1950 congress recognized puerto rico s authority over internal matters through passage of the federal relations act congress also approved puerto rico s constitution in 1952 so we gave them the control they demanded and with that they attempted to become a liberal paradise by raising taxes expanding government programs and spending at unsustainable rates to help pay for these policies puerto rico issued billions of dollars in bonded debt that they can no longer pay back now they are demanding help which puts congress in a very difficult position the fact that we have reached this point is a direct result of the president and the treasury department being asleep at the switch they either were not paying attention to the financial situation in puerto rico or they were paying attention and chose to do nothing i want to highlight a few important things about this bill first this bill is not a bailout the american taxpayers did not create this problem and we shouldn t send their money to something they did not cause what really worries me is that if congress doesn t act on this legislation then we will find ourselves in a position at some point facing serious pressure to vote on a true actual bailout of puerto rico that would be a grave mistake as the president of americans for tax reform noted in an op ed for the national review congress needs to step in now otherwise a huge taxpayer bailout is the likely outcome promesa is the best most fiscally responsible way to prevent a bailout from occurring this bill does not include a single penny in taxpayer money in fact the congressional budget office found that this bill would have no significant net effect on the federal deficit so let s try and get this problem resolved in a fiscally responsible way that does not use taxpayer dollars second the policies in puerto rico have led to this problem so it is important that the legislation address some of these policies and require greater accountability the bill does this through the creation of a seven person financial oversight board which is responsible for the development of budgets and fiscal plans for puerto rico the bill also includes some commonsense policy changes that will hopefully ease the burdens on the puerto rican government by prohibiting the costly new overtime rule from taking effect and giving them flexibility with minimum wage requirements for young workers through better oversight and regulatory reforms it is my belief the puerto rican economy can grow and the country can get back on a more stable financial footing i want to make one thing very clear i and every member of this house have great empathy and appreciation for the puerto rican people because they did not cause this problem i have had the honor of traveling to puerto rico and visiting this beautiful place i enjoyed meeting the people and really appreciated their hospitality i believe it is important we do what we can in a responsible manner to support the puerto rican people ultimately i wish this legislation wasn t necessary but the reality of the situation demands action so i call on my colleagues to support this rule support the underlying bill and let s address this problem in a responsible way without a bailout mr speaker i reserve the balance of my time
\vspace{8mm}
recent iranian ballistic missile tests in direct violation of sanctions by the united nations show that this regime cannot be trusted this week a panel of experts from the united nations confirmed that the tests in october and november violated sanctions placed on iran in june of 2010 the tests also stand in stark contrast to the joint comprehensive plan of action the agreement unveiled by president barack obama which is intended to curb the iranian nuclear program this is a plan which would roll back sanctions against the regime at a time when the united nations security council is considering new sanctions due to these missile tests the idea that we should reward iran by removing economic sanctions providing billions to a regime that has long been the leading state sponsor of terrorism is dangerous past performance is a good indication of future actions iran has a decades long history of misrepresentation to the global community especially in regards to its nuclear program i urge the president to abandon the joint comprehensive plan of action in {\bf \color{red} order} to make sure not 1 flows into the coffers of this terrorist regime\pagebreak

\end{document}